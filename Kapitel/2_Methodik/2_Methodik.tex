\newpage
\section{Methodik}

Diese Bachelorarbeit befasst sich mit systematischen Produktentwicklung eines Pflanzroboters. Das systematische Vorgehen orientiert sich an der VDI-Richtlinie 2221, welche in folgende Phasen unterteilt wird:

\begin{itemize}
	\item \textbf{Phase I:} Planen und klären der Aufgabe durch informative Festlegung
	
	\item \textbf{Phase II:} Konzipieren durch prinzipielle Festlegung
	
	\item \textbf{Phase III:} Gestalterische Festlegung der angestrebten Lösung
	
	\item \textbf{Phase IV:} Ausarbeiten, Erstellung der erforderlichen Unterlagen
	
\end{itemize}

Phase I stellt der eigentliche Entwurf des Systems dar. Nach VDI-Richtlinie 2221 (\textit{Quelle}) beinhaltet diese Phase im Wesentlichen das Klären der Aufgabenstellung sowie die Ermittlung von Funktionen. Für die Klärung der Aufgabenstellung wird in Zusammenarbeit mit dem Auftraggeber (MCC Laboratoire Meiners GmbH) ein Pflichtenheft ausgearbeitet. Darin sind alle relevanten Eigenschaften und Anforderungen an das Produkt formuliert. Dieses Dokument dient als Grundlage für die weitere Entwicklung des Produktes und grenzt die Aufgabenstellung der Hochschule weiter ein. Der zweite Teil von Phase I, die Ermittlung der Funktionen, wurde nach Pahl und Beitz (2003) angelehnt:

\begin{itemize}
	\item \textbf{Hauptfunktion:} Basierend auf der Aufgabenstellung sowie dem erarbeitetem Pflichtenheft ergibt sich die Hauptfunktion.
	
	\item \textbf{Aufgliederung in Teilfunktionen:} Die Hauptfunktion wird mittels Funktionsanalyse in verschiedene Teilfunktionen unterteilt. So wird die Aufgabenstellung abstrahiert und rein funktional betrachtet. Dadurch wird eine möglichst lösungsneutrale Betrachtung des Problems gewährleistet. 
	
	\item \textbf{Funktionsbezogene Variation:} Für die ausgearbeiteten Teillösungen werden mögliche Lösungsansätze erarbeitet. Dafür wurden Kreativitätstechniken verwendet, um pro Teilfunktion eine Vielzahl von Lösungsansätzen zu generieren.

\end{itemize}

Phase II nach  VDI-Richtlinie 2221 befasst sich mit der Ausarbeitung eines Konzepts. Dabei werden die generierten Ideen aus der Funktionsbezogenen Variation aufgegriffen und zu einem Konzept ausgearbeitet. Aufgebaut ist diese Phase wie folgt:

\begin{itemize}
	\item \textbf{Suchen nach Lösungsprinzipien:} konkrete Teillösungen der funktionsbezogenen Variation werden zu einer Gesamtlösung zusammengesetzt. So entsteht ein Gesamtkonzept. Dafür wurden alle Teillösungen in einem morphologischen Kasten aufgelistet und in einem ersten Schritt fünf Konzepte ausgearbeitet. In einem zweiten Schritt wurden die Auswahl der Konzepte auf drei eingegrenzt (wie?).
	
	\item \textbf{Gliedern in realisierbare Module:} Die Konzepte wurden durch die Gestaltung von Konzeptskizzen visualisiert. 
	
	\item \textbf{Funktionsnachweis kritischer Funktionen:} Ergänzend zur VDI-Richtlinie 2221 wurde ein Funktionsnachweis von kritischen Funktionen erbracht. Dabei wurden einfache aufgebaute, praktische Versuche der erarbeiteten Teillösung realisiert, um die Funktionalität zu validieren.
	
	\item \textbf{Bewertung:} Hinzukommend zur VDI-Richtlinie 2221 wurden die ausgearbeiteten Konzepte einer objektiven Bewertung unterzogen. Dabei wurde eine Nutzwertanalyse nach Pahl und Beitz (2003) durchgeführt:
	\begin{itemize}
		\item Fünf Ziele wurden definiert, welche funktionale sowie auch wirtschaftliche Anforderungen beinhalten. Für jedes Ziel wurde eine Gewichtung vorgenommen, wobei die Funktionserfüllung höher gewichtet wurde als der Kostenaspekt.
		
		\item Jedes der ausgearbeiteten Konzepte wurde nach einer Wertskala von 0 - 4 (von unbefriedigend bis ideal, gemäss VDI-Richtlinie 2225) bewertet. Dabei wurden auch die gewonnenen Erkenntnisse aus dem Funktionsnachweis miteinbezogen. Wichtig ist dabei, dass die Beurteilung völlig objektiv erfolgt.
	\end{itemize}	
	
	\item \textbf{Entscheid:} Die Nutzwertanalyse wird ausgewertet und das bestbenotete Konzept ausgewählt. Dieses kommt nun in Phase III zur Umsetzung. 
\end{itemize}

In Phase III wird das ausgewählte Konzept umgesetzt. Dabei wird in dieser Phase vermehrt fachspezifisch gearbeitet. Aufgebaut ist die Umsetzungsphase folgendermassen:
\begin{itemize}
	\item \textbf{Gestalten der Module:} Die Teillösungen des Konzepts werden konkret in Baugruppen realisiert. Dies geschieht auf allen technischen Ebenen; sei es Software, PCB-Design oder Konstruktion. Zuerst wird dabei eine Evaluation von Komponenten durchgeführt. Wenn nötig, wurden Berechnungen zur Auswahl der Komponenten gemacht. Dabei ist eine stetige Absprache zwischen Elektrotechnik sowie Maschinentechnik essentiell, um Missverständnisse und Fehler zu vermeiden.
	
	\item \textbf{Gestalten des Produktes:} Dies beinhaltet das endgültige Zusammenführen aller Baugruppen zu einem Produkt. Fertigungsunterlagen für die verwendeten Fertigungsverfahren werden erstellt und in Auftrag gegeben. Zudem wird die Bestellung der Komponenten ausgelöst und Offerten von Herstellern eingeholt. 
\end{itemize}

Die Phase IV umfasst die Ausarbeitung des Produktes. Dabei weicht die Bachelorarbeit von der VDI-Richtlinie 2221 ab, da sich diese Arbeit mit der Entwicklung eines Funktionsmusters befasst (und nicht eines Serienproduktes). Dabei wird in dieser Phase auch eine Ausarbeitung betrieben, jedoch mit dem klaren Fokus der Montage sowie Inbetriebnahme des Funktionsmusters. Die einzelnen Baugruppen werden an Funktionstests unterzogen dadurch auf ihre Funktionalität geprüft. Sind alle Baugruppen funktionsfähig, kann das Gesamtsystem als Ganzes getestet werden.
\newline
Während der gesamten Bachelorarbeit werden alle relevanten Schritte und Überlegungen (methodischer sowie technischer Art) dokumentiert und in einem Bericht festgehalten.