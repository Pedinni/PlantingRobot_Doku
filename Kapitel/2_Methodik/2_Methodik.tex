\newpage
\section{Methodik}

Diese Bachelorarbeit befasst sich mit systematischen Produktentwicklung eines Pflanzroboters. Das systematische Vorgehen orientiert sich an der VDI-Richtlinie 2221, welche in folgende Phasen unterteilt wird:

\begin{itemize}
	\item \textbf{Phase I:} Planen und klären der Aufgabe durch informative Festlegung
	
	\item \textbf{Phase II:} Konzipieren durch prinzipielle Festlegung
	
	\item \textbf{Phase III:} Gestalterische Festlegung der angestrebten Lösung
	
	\item \textbf{Phase IV:} Ausarbeiten, Erstellung der erforderlichen Unterlagen
	
\end{itemize}

Phase I stellt der eigentliche Entwurf des Systems dar. Nach VDI-Richtlinie 2221 (\textit{Quelle}) beinhaltet diese Phase im Wesentlichen das Klären der Aufgabenstellung sowie die Ermittlung von Funktionen. Für die Klärung der Aufgabenstellung wird in Zusammenarbeit mit dem Auftraggeber (MCC Laboratoire Meiners GmbH) ein Pflichtenheft ausgearbeitet. Darin sind alle relevanten Eigenschaften und Anforderungen an das Produkt formuliert. Dieses Dokument dient als Grundlage für die weitere Entwicklung des Produktes und grenzt die Aufgabenstellung der Hochschule weiter ein. Der zweite Teil von Phase I, die Ermittlung der Funktionen, wurde nach Pahl und Beitz (2003) durchgeführt:

\begin{itemize}
	\item \textbf{Hauptfunktion:} Basierend auf der Aufgabenstellung sowie dem erarbeitetem Pflichtenheft ergibt sich die Hauptfunktion.
	
	\item \textbf{Aufgliederung in Teilfunktionen:} Die Hauptfunktion wird unterteilt in verschiedene Teilfunktionen. Mittels Funktionsanalyse wird die Aufgabenstellung abstrahiert und rein funktional betrachtet. Dadurch wird eine möglichst lösungsneutrale Betrachtung des Problems gewährleistet. 
	
	\item \textbf{Funktionsbezogene Variation:} Gestalterische Festlegung der angestrebten Lösung
	
	\item \textbf{Phase IV:} Ausarbeiten, Erstellung der erforderlichen Unterlagen
	
\end{itemize}