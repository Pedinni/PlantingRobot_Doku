\newpage
\section{Methodik}
\subsection{Vorgehen}
\textit{(ygu)} Diese Bachelorarbeit befasst sich mit systematischen Produktentwicklung eines Pflanzroboters. Das systematische Vorgehen orientiert sich  an der VDI-Richtlinie 2221, welche in folgende Phasen unterteilt wird \cite{naefe}:

\begin{itemize}
	\item \textbf{Phase I:} Planen und klären der Aufgabe durch informative Festlegung
	
	\item \textbf{Phase II:} Konzipieren durch prinzipielle Festlegung
	
	\item \textbf{Phase III:} Gestalterische Festlegung der angestrebten Lösung
	
	\item \textbf{Phase IV:} Ausarbeiten, Erstellung der erforderlichen Unterlagen
	
\end{itemize}

Phase I stellt der eigentliche Entwurf des Systems dar. Gemäss VDI-Richtlinie 2221 beinhaltet diese Phase im Wesentlichen das Klären der Aufgabenstellung sowie die Ermittlung von Funktionen \cite{vdi2221}. Für die Klärung der Aufgabenstellung wurde in Zusammenarbeit mit dem Auftraggeber (MCC Laboratoire Meiners GmbH) ein Pflichtenheft ausgearbeitet. Darin sind alle relevanten Eigenschaften und Anforderungen an das Produkt formuliert. Dieses Dokument dient als Grundlage für die weitere Entwicklung des Produktes und grenzt die Aufgabenstellung der Hochschule weiter ein. Der zweite Teil von Phase I, die Ermittlung der Funktionen, gestaltet sich folgendermassen \cite{pahl}:

\begin{itemize}
	\item \textbf{Hauptfunktion:} Auf der Aufgabenstellung sowie dem erarbeitetem Pflichtenheft aufbauend ergibt sich die Hauptfunktion des Pflanzroboters.
	
	\item \textbf{Aufgliederung in Teilfunktionen:} Die Hauptfunktion wird mittels Funktionsanalyse in verschiedene Teilfunktionen unterteilt. So wird die Aufgabenstellung abstrahiert und rein funktional betrachtet. Dadurch wird eine möglichst lösungsneutrale Betrachtung des Problems gewährleistet. 
	
	\item \textbf{Funktionsbezogene Variation:} Für die ausgearbeiteten Teillösungen werden mögliche Lösungsansätze erarbeitet. Dafür wurden Kreativitätstechniken verwendet, um eine Vielzahl von Lösungsansätzen für jede Teilfunktion zu generieren.

\end{itemize}

Nach  VDI-Richtlinie 2221 befasst sich Phase II mit der Ausarbeitung eines Konzepts. Dabei werden die generierten Ideen aus der funktionsbezogenen Variation aufgegriffen und zu einem Konzept ausgearbeitet. Aufgebaut ist diese Phase wie folgt \cite{vdi2221}:

\begin{itemize}
	\item \textbf{Suchen nach Lösungsprinzipien:} konkrete Teillösungen der funktionsbezogenen Variation werden zu einer Gesamtlösung zusammengesetzt. So entsteht ein Gesamtkonzept. Dafür wurden alle Teillösungen in einem morphologischen Kasten aufgelistet und in einem ersten Schritt fünf Konzepte ausgearbeitet. In einem weiteren Schritt wurden die Auswahl auf drei Konzepte eingegrenzt (wie?).
	
	\item \textbf{Gliedern in realisierbare Module:} Die Konzepte wurden durch die Gestaltung von Konzeptskizzen visualisiert und zum ersten Mal als gesamtes Konzept dargestellt. 
	
	\item \textbf{Funktionsnachweis kritischer Funktionen:} Ergänzend zur VDI-Richtlinie 2221 wurde ein Funktionsnachweis von kritischen Funktionen erbracht. Dabei wurden einfache aufgebaute, praktische Versuche der erarbeiteten Teillösung realisiert, um die Funktionalität zu validieren (überprüfen?).
	
	\item \textbf{Bewertung:} Hinzukommend zur VDI-Richtlinie 2221 wurden die ausgearbeiteten Konzepte einer objektiven Bewertung unterzogen. Dabei wurde eine Nutzwertanalyse durchgeführt \cite{pahl}:
	\begin{itemize}
		\item Fünf Ziele wurden definiert, welche funktionale sowie auch wirtschaftliche Anforderungen beinhalten. Für jedes Ziel wurde eine Gewichtung vorgenommen, wobei die Funktionserfüllung höher gewichtet wurde als der Kostenaspekt.
		
		\item Jedes der ausgearbeiteten Konzepte wurde nach VDI-Richtlinie 2225 mit einer Wertskala von 0 bis 4 (von unbefriedigend bis ideal) bewertet \cite{vdi2225}. Dabei wurden auch die gewonnenen Erkenntnisse aus dem Funktionsnachweis miteinbezogen. Wichtig ist dabei, dass die Beurteilung völlig objektiv erfolgt.
	\end{itemize}	
	
	\item \textbf{Entscheid:} Die Nutzwertanalyse wird ausgewertet und das Konzept mit dem höchsten Nutzwert ausgewählt. Dieses kommt nun in Phase III zur Umsetzung. 
\end{itemize}

In Phase III wird das ausgewählte Konzept umgesetzt. Dabei wird in dieser Phase vermehrt fachspezifisch gearbeitet. Aufgebaut ist die Umsetzungsphase folgendermassen:
\begin{itemize}
	\item \textbf{Gestalten der Module:} Die Teillösungen des Konzepts werden konkret in Baugruppen realisiert. Dies geschieht auf allen technischen Ebenen; sei es Software, PCB-Design oder Konstruktion. Zuerst wurde hierfür eine Evaluation von möglichen Komponenten durchgeführt. Wenn nötig, wurden Berechnungen zur Auswahl der Komponenten gemacht. Dabei ist eine stetige Absprache zwischen Elektrotechnik und Maschinentechnik essentiell, um Missverständnisse und Fehler zu vermeiden.
	
	\item \textbf{Gestalten des Produktes:} Dies beinhaltet das konstruktive Zusammenführen aller Baugruppen zu einem Produkt. Fertigungsunterlagen für die verwendeten Fertigungsverfahren wurden erstellt und in Auftrag gegeben. Parallel dazu wurde die Bestellung der Komponenten ausgelöst und Offerten von Herstellern eingeholt. 
\end{itemize}

Die Phase IV umfasst die Ausarbeitung des Produktes. Dabei weicht die Bachelorarbeit von der VDI-Richtlinie 2221 ab, da sich diese Arbeit mit der Entwicklung eines Funktionsmusters befasst (und nicht eines Serienproduktes). Die Ausarbeitung in dieser Phase befasst sich mit dem klaren Fokus der Montage sowie Inbetriebnahme des Funktionsmusters. Die einzelnen Baugruppen werden an Funktionstests unterzogen dadurch auf ihre Funktionalität geprüft. Sind alle Baugruppen funktionsfähig, kann das Gesamtsystem als Ganzes getestet werden.
\newline
Während der gesamten Bachelorarbeit werden alle relevanten Schritte und Überlegungen (methodischer sowie technischer Art) dokumentiert und in einem Bericht festgehalten.