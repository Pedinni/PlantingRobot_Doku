\newpage
\section{Konzept}
\label{konzept}
\textit{(ygu)} Dieses Kapitel befasst sich mit der Ausarbeitung und Bewertung von drei möglichen Lösungskonzepten. Die Auswahl dieser Konzepte basiert auf der Anwendung des morphologischen Kastens, der als Methode von Zwicky begründet wurde (\citeNP[S.~73]{naefe}).   In einer Tabelle werden alle gefunden Teillösungen nach Teilfunktion aufgelistet. Eine Gesamtlösung (Lösungskonzept) ergibt sich aus der Kombination von je einem Element pro Teilfunktion (Teillösung). Die Verträglichkeit der verschiedenen Elemente wird durch die Selektion miteinbezogen \cite{naefe}.
\newline

Die umfassende Anzahl von möglichen Kombinationen ist ein Vorteil, birgt jedoch auch Schwierigkeiten. So kann eine zu hohe Anzahl von Kombinationen die zeitlichen Rahmenbedingungen der vorliegenden Bachelorarbeit rasch sprengen. Aus diesem Grund beschränkt sich die Auswahl auf drei Konzepte. Für diesen wiederholenden Prozess wird die Methode des „Ausscheidens und Bevorzugens“ verwendet \cite{naefe}. Weiter werden die möglichen Kombinationen durch das Pflichtenheft eingeschränkt. Dabei werden ungeeignete Kombinationen ausgeschlossen, Kombinationen die den Fest- und Wunschanforderungen entsprechen werden bevorzugt.
\newline

Aus diesem Prozess sind drei Lösungskonzepte entstanden. Im Anschluss werden sie prägnant beschrieben und einer kritischen Beurteilung unterzogen. Am Ende der Konzeptphase wird entschieden, welches der drei Konzepte als Funktionsmuster realisiert wird.
