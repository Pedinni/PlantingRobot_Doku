\newpage
\section{Konzept}
Dieses Kapitel befasst sich mit der Ausarbeitung von drei möglichen Lösungskonzepten und deren Bewertung. Für die Auswahl der drei Lösungskonzepte wird die Methode des morphologischen Kastens nach Zwicky angewandt. In einer Tabelle werden alle gefunden Teillösungen nach Teilfunktion aufgelistet. Eine Gesamtlösung (Lösungskonzept) ergibt sich aus der Kombination von je einem Element pro Teilfunktion (Teillösung). Nach Neafe (2015) ist bei der selektiven Wahl der Elemente auf die Verträglichkeit der Elemente untereinander zu achten.
\newline
Die schiere Anzahl von möglichen Kombinationen ist ein Vorteil, kann jedoch auch überborden. Eine zu hohe Anzahl Kombinationen kann rasch zu einem Arbeitsaufwand führen, der nicht im gesetzten Zeitrahmen dieser Bachelorarbeit zu bewältigen wäre. Dieser Versuchung wird entgegnet indem sich die Auswahl auf drei Konzepte beschränkt. Weiter wird die Auswahl durch das Pflichtenheft eingeschränkt. Ungeeignete Kombinationen sind auszuschliessen, Kombinationen die den Fest- und Wunschanforderungen entsprechen sind zu bevorzugen.

Aus diesem Prozess sind drei Lösungskonzepte entstanden. Diese werden kurz vortstellt und einer Beurteilung unterzogen. Orientiert an der Beurteilung, steht am Ende der Konzeptphase der Entscheid, welches der drei Konzepte als Funktionsmuster umgesetzt wird.
