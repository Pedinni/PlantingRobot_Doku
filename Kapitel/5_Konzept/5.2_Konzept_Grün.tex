\newpage
\subsection{Konzept Grün}
\label{KonzeptGreen}
\textit{(ygu)} Konzept Grün besteht aus folgenden Teilen:

\subsubsection{Vereinzelung durch Wendelförderer}

Das Konzept Grün beinhaltet die Vereinzelung sowie Förderung von NemaCaps mit einem Vibrationswendelförderer (siehe Abbildung \ref{fig:vereinzelung_green}).
Vibration ist für die Vereinzelung sowie Förderung von Gütern mit einfacher Geometrie eine oft genutzte Technik. Dabei wird mittels Schwingungsenergie (erzeugt durch eine Unwuchterregung mittels Vibrator) die zu bewegende Masse erregt und eine Hin- und Rückbewegung erzeugt \cite{risch}. Die Masse wird so angestossen, dass eine Kette von Wurfbewegungen entsteht und sich die Masse fortbewegt (siehe Abbildung \ref{fig:foerderbewegung}).


\begin{minipage}{0.475\textwidth}	
	\includegraphics[width=0.85\textwidth]{Illustrationen/5-Konzept/foerderbewegung.png}
	\captionof{figure}{Wurfbewegung erzeugt durch Vibration \protect\cite{webac}}
	\label{fig:foerderbewegung}
\end{minipage}
\hfill
\begin{minipage}{0.475\textwidth}
	\includegraphics[width=\textwidth]{Illustrationen/5-Konzept/schikane.png}
	\captionof{figure}{Vereinzelung durch Schikanen \protect\cite{handling_online}}
	\label{fig:schikane}
\end{minipage}	

\vspace{0.6cm}

Die Vereinzelung wird mit Schikanen realisiert. Diese gewährleisten durch Abstimmung der Werkstückkontur, dass jeweils nur ein Werkstück die Schikane passiert und weiter gefördert wird. Beispiele solcher Schickanen sind Details 1, 2 und 3 in Abbildung \ref{fig:schikane}. Überzählige werden zurück ins Haufwerk gelenkt \cite{handling_online}.
\newline
Der vorgesehene Wendelförderer basiert auf den erwähnten Techniken. In einer Spirale werden durch die Vibration die Werkstücke gefördert und sogleich vereinzelt. Speziell an dieser Anwendung ist, dass die NemaCaps in drei parallele Bahnen geteilt werden. Auch denkbar ist die Verwendung dreier einzelner Wendelförderer.

\begin{figure}[H]
	\includegraphics[scale=0.5]{Illustrationen/5-Konzept/green_wendelfoerderer.jpg}
	\caption{Konzeptskizze I Konzept Grün: Wendelförderer}
	\label{fig:vereinzelung_green}
\end{figure}

\subsubsection{Setzen mittels Pick-and-Place Bewegung}
Die NemaCaps kommen geordnet bei der Setzeinheit an. Dabei werden diese an einem mechanischen Anschlag gestoppt (Siehe Punkt \textbf{1} in Abbildung \ref{fig:transport_green_side}). Von dort werden die NemaCaps durch eine Setzeinheit erfasst und in den Töpfen platziert. Dabei ist die Setzeinheit als zweidimensionale Pick-and-Place Maschine aufgebaut. Diese zwei Dimensionen sind:
\begin{itemize}
	\item Translation in X-Richtung: Diese Bewegung dient zum horizontalen Transport von mechanischem Anschlag zum Topf.
	\item Translation in Z-Richtung: Um das NemaCap zu Packen sowie im Topf zu platzieren wird diese Bewegung ausgeführt.
\end{itemize}
Zur Übersicht dient Abbildung \ref{fig:transport_green_pers}. 
\begin{figure}[H]
	\includegraphics[scale=0.6]{Illustrationen/5-Konzept/green_2Dmachine_pervsp.jpg}
	\caption{Konzeptskizze II Konzept Grün: Perspektivische Ansicht der Pick-and-Place Bewegung}
	\label{fig:transport_green_pers}
\end{figure}

Der Prozess der Setzeinheit gestaltet sich wie folgt:
\begin{itemize}
	\item \textbf{A - NemaCap packen:} Sobald der bewegte Teil der Seitzeinheit sich über dem NemaCap befindet, packen drei Dorne je ein NemaCap (Detail \textbf{A} in Abbildung \ref{fig:transport_green_side}). Für das Packen sind folgende Techniken möglich:
	\begin{itemize}
		\item Mittels Zange wird das NemaCap seitlich gefasst.
		\item Durch einen spitzen Dorn wird das NemaCap aufgespiesst (Gemäss Pflichtenheft zulässig).
		\item Durch einen stumpfen Dorn mit hoher Adhäsion (zum Beispiel Kleber) wird das NemaCap angeheftet.
		\item Mittels Unterdruck wird das NemaCap am Dorn angesaugt.
	\end{itemize} 
	\item \textbf{B - NemaCap transportieren:} Das NemaCap wird durch die Bewegung in X-Richtung über die Einsatzlokalität transportiert. Dabei werden die einzelnen Dorne durch Laufschienen geleitet und dadurch auf den entsprechenden Topfradius gelenkt. Die einzelnen Dorne sind miteinander verbunden, sodass diese lineare Bewegung nur einen Aktor erfordert.
	
	\item \textbf{C - NemaCap setzen:} 
	Über der Einsetzlokalität angekommen, bewegt sich der Dorn in Z-Richtung nach unten und stösst das NemaCap in die Erde. Dabei wird eine Beschädigung des NemaCaps bewusst in Kauf genommen. Dies ist gemäss Pflichtenheft zulässig. Wichtig ist, dass sich das NemaCap beim Zurückfahren des Dorns löst.
\end{itemize}
Die Konfiguration des Setzmechanismus ist durch die Verstellung der Laufschienen möglich. Dabei sind die Laufschienen (Punkt 1 in Abbildung \ref{fig:transport_green_pers}) am oberen Ende (über dem Topf) in Y-Richtung verstellbar. So können die verschiedenen Topfgrössen gehandhabt werden.

\begin{figure}[H]
	\includegraphics[scale=0.6]{Illustrationen/5-Konzept/green_2Dmachine_seite.jpg}
	\caption{Konzeptskizze III Konzept Grün: Seitenansicht der Pick-and-Place Bewegung}
	\label{fig:transport_green_side}
\end{figure}