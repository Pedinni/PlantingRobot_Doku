\subsection{Elektronik}

Zur Übersicht der verwendeten Elektronik Hardware dient das Blockschaltbild Abb. \ref{fig:Blockschaltbild_Komponenten}. Es bildet eine detaillierte Auflistung sämtlicher Elektronik Komponenten. Wie im Blockschaltbild dargestellt werden sämtliche Elektronikkomponenten über das selbst entwickelte Mainboard PCB miteinander verknüpft. Auf dem Mainboard PCB befindet sich das Mikrocontroller Board FRDM-KL25Z, auf welchem ein RTOS mit entsprechender Software gemäss Kap. \ref{kap:Software} implementiert ist.

\begin{figure}[H]
	\includegraphics[width=1\textwidth]{Illustrationen/5-Konzept/Blockschaltbild_Komponenten.png}
	\caption{Blockschaltbild der Elektronik Komponenten}
	\label{fig:Blockschaltbild_Komponenten}
\end{figure}

Im folgenden Abschnitt wird das Blockschaltbild ausführlich erklärt:
\begin{itemize}
	\item \textbf{Netzbeschaltung:} Gemäss Pflichtenheft, soll der Planting Robot über 230V Netzspannung betrieben werden können. Diese Anforderung setzt eine adäquate Schutzschaltung voraus. So wurde ein Leitungsschutzschalter, sowie eine Relais Selbsthalteschaltung zum Ein und Ausschalten der Anlage über zwei Taster mit Betriebsanzeige vorgesehen. Des weiteren wurden sämtliche Gehäuseteile aus Metall, innerhalb des Schaltschrankes, mit dem Schutzleiter elektrisch verbunden.

	\item \textbf{36V Netzteil:} Um die hohe Leistungsanforderung an den Spindelantrieb bei moderatem Stromfluss erfüllen zu können, wird dessen Motor mit einem 36V Netzteil gespeist. Das FRDM-Board kann über das enable Signal des Netzteils die 36V Ausgangsspannung ein- und ausschalten.
	
	\item \textbf{BLDC-Controller:} Der BLDC-Controller wird über einen UART Port des FRDM-Boards angesteuert. Mit Hilfe der Encoder Auswertung des BLDC-Controllers kann der BLDC-Motor als Servomotor mit Positionsregelung betrieben werden. Damit kann ein Maximum an Präzision und Geschwindigkeit bei der Ansteuerung des Spindelantriebs garantiert werden.
	
	\item \textbf{Baseboard PCB:} Als Schnittstelle zwischen Mainboard PCB und BLDC-Controller dient das Baseboard PCB. Es vereinfacht die Verkabelung mit dem BLDC-Controller Board und verfügt über Drehpotentiometer zur schnellen Inbetriebnahme des BLDC-Motors.
	
	\item \textbf{Spindelantrieb:} Für eine maximale Lebensdauer bei häufigen Lastwechseln und ein hohes Anzugdrehmoment wird ein BLDC-Motor mit Hall Sensoren verwendet.
			
	\item \textbf{12V Netzteil:} Um einen sicheren Betrieb der Elektronik während hohen Leistungsspitzen des Spindelantriebs zu garantieren, wird diese über ein separates Netzteil versorgt. Des weiteren werden auch die kleineren DC Getriebemotoren über dieses Netzteil gespeist. Das 12V Netzteil wird über das Zuschalten der Netzspannung gesteuert und kann nicht über den Mikrocontroller geschaltet werden.
\end{itemize}