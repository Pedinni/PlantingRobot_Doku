\subsection{Bewertung}
Für die Bewertung der Konzepte wird eine Nutzwertanalyse durchgeführt \cite{naefe}. Angepasst auf diese Bachelorarbeit ergibt sich folgendes Vorgehen:

\begin{enumerate}
	\item \textbf{Erkennen der Bewertungskriterien:} Für die objektive Bewertung der Konzepte werden Bewertungskriterien definiert.
	
	\item \textbf{Gewichtung der Kriterien:} Für jedes Kriterium wird ein Gewichtungsfaktor festgelegt, um die Kriterien nach ihrer Wichtigkeit einzuordnen.
	
	\item \textbf{Beurteilen nach Wertvorstellung:} Mit der definierten Wertskala nach VDI-Richtlinie 2225 werden die einzelnen Konzepte an den definierten Kriterien bewertet \cite{vdi2225}.

\begin{table}[H]
	\begin{tabular}{|c|l|}
		\hline 
		\textbf{Punkte} & \textbf{Bedeutung} \\ 
		\hline 
		0 & unbefriedigend \\ 
		\hline 
		1 & gerade noch tragbar \\ 
		\hline 
		2 & ausreichend \\ 
		\hline 
		3 & gut \\ 
		\hline 
		4 & sehr gut (ideal) \\ 
		\hline 
	\end{tabular} 
	\caption{Wertskala nach VDI-Richtlinie 2225}
	\label{tab:wertskala}
\end{table}	
	
	
	\item \textbf{Bestimmung des Nutzwertes:} Aus der Bewertung wird der Nutzwert von jedem Konzept ermittelt. Das bestbewertete Konzept wird umgesetzt.
\end{enumerate}

\subsubsection{Definition und Gewichtung der Kriterien}
Bei der Definition der Kriterien ist darauf zu achten, dass jedes Kriterium Relevanz für die Aufgabenstellung hat. Als Orientierung wurde die "Leitlinie mit Hauptmerkmalen zum Bewerten in der Konzeptionsphase" verwendet \cite{pahl}. 
\newline
Die Kriterien bestehen aus:
\begin{itemize}
	\item Komplexität
	\begin{itemize}
		\item Geringer Grad der technischen Komplexität
		
		\item gebräuchliche Fertigungsverfahren, keine aufwendige Vorrichtungen, einfach gestaltete Teile
		
		\item leichte und schnelle Montage
	\end{itemize}

	\item Risiken
	\begin{itemize}
	\item Funktionserfüllung mit ausreichender Wirkung und geringen Störgrössen
	
	\item Hohe Zuverlässigkeit während dem Betrieb
	
	\item abschätzbare Risiken in der Umsetzung
	\end{itemize}

	\item Kosten
	\begin{itemize}
	\item geringe Kosten Investitionskosten
	
	\item geringe Betriebs- und Unterhaltskosten
	\end{itemize}

	\item Verfügbarkeit der Komponenten
	\begin{itemize}
	\item Komponenten sind in nützlicher Frist lieferbar
	
	\item Verwendung von Normteilen
	
	\item Alternativprodukte sind verfügbar
	\end{itemize}

	\item Innovationsgrad
	\begin{itemize}
	\item Innovatives Gesamtkonzept
	
	\item einfache Funktionserfüllung
	
	\item pragmatische Kombination von Teilfunktionen
	\end{itemize}

	\item Benutzerfreundlichkeit
	\begin{itemize}
	\item einfache und intuitive Handhabung im Betrieb	
		
	\item Mensch-Maschinen-Beziehung zufriedenstellend
	
	\item geringer Unterhalt
	
	\end{itemize}

\end{itemize}

Nach folgender Gewichtung werden die Konzepte dabei bewertet:
\begin{table}[H]
	\begin{tabular}{|l|c|}
	\hline 
	\textbf{Kriterium} & \textbf{Gewichtung} \\ 
	\hline 
	Komplexität & 0.25 \\ 
	\hline 
	Risiken & 0.25 \\ 
	\hline 
	Kosten & 0.05 \\ 
	\hline 
	Verfügbarkeit der Komponenten & 0.2 \\ 
	\hline 
	Innovationsgrad & 0.05 \\ 
	\hline 
	Benutzerfreundlichkeit & 0.2 \\ 
	\hline 
	\end{tabular} 
	\caption{Gewichtung der Bewertungskriterien}
	\label{tab:gewichtung}
\end{table}