\subsection{Bewertung und Entscheid}
\textit{(ygu)} Für die Bewertung der Konzepte wird eine Nutzwertanalyse durchgeführt \cite{naefe}. Angepasst auf die vorliegende Bachelorarbeit ergibt sich folgendes Vorgehen:

\begin{enumerate}
	\item \textbf{Erkennen der Bewertungskriterien:} Für die objektive Bewertung der Konzepte werden Bewertungskriterien definiert.
	
	\item \textbf{Gewichtung der Kriterien:} Für jedes Kriterium wird ein Gewichtungsfaktor festgelegt, um die Kriterien nach ihrer Wichtigkeit einzuordnen.
	
	\item \textbf{Beurteilen nach Wertvorstellung:} Mit der definierten Wertskala nach VDI-Richtlinie 2225 werden die einzelnen Konzepte an den definierten Kriterien bewertet \cite{vdi2225}.

\begin{table}[H]
	\begin{tabular}{|c|l|}
		\hline 
		\textbf{Punkte} & \textbf{Bedeutung} \\ 
		\hline 
		0 & unbefriedigend \\ 
		\hline 
		1 & gerade noch tragbar \\ 
		\hline 
		2 & ausreichend \\ 
		\hline 
		3 & gut \\ 
		\hline 
		4 & sehr gut (ideal) \\ 
		\hline 
	\end{tabular} 
	\caption{Wertskala nach VDI-Richtlinie 2225}
	\label{tab:wertskala}
\end{table}	
	
	
	\item \textbf{Bestimmung des Nutzwertes:} Aus der Bewertung wird der Nutzwert von jedem Konzept ermittelt. Das bestbewertete Konzept wird umgesetzt.
\end{enumerate}

\subsubsection{Definition und Gewichtung der Kriterien}
\textit{(ygu)} Bei der Definition der Kriterien ist darauf zu achten, dass jedes einzelne Kriterium relevant für die grundlegende Aufgabenstellung ist.
Als Orientierung wurde die „Leitlinie mit Hauptmerkmalen zum Bewerten in der Konzeptionsphase“ verwendet \cite{pahl}. 
\newline
Die Kriterien beinhalten folgende Punkte:
\begin{itemize}
	\item Komplexität
	\begin{itemize}
		\item Geringer Grad der technischen Komplexität
		
		\item gebräuchliche Fertigungsverfahren, keine aufwendige Vorrichtungen, einfach gestaltete Teile
		
		\item leichte und schnelle Montage
	\end{itemize}

	\item Risiken
	\begin{itemize}
	\item Funktionserfüllung mit ausreichender Wirkung und geringen Störgrössen
	
	\item Hohe Zuverlässigkeit während dem Betrieb
	
	\item abschätzbare Risiken in der Umsetzung
	\end{itemize}

	\item Kosten
	\begin{itemize}
	\item geringe Investitionskosten
	
	\item geringe Betriebs- und Unterhaltskosten
	\end{itemize}

	\item Verfügbarkeit der Komponenten
	\begin{itemize}
	\item Komponenten sind in nützlicher Frist lieferbar
	
	\item Verwendung von Normteilen
	
	\item Alternativprodukte sind verfügbar
	\end{itemize}

	\item Innovationsgrad
	\begin{itemize}
	\item Innovatives Gesamtkonzept
	
	\item einfache Funktionserfüllung
	
	\item pragmatische Kombination von Teilfunktionen
	\end{itemize}

	\item Benutzerfreundlichkeit
	\begin{itemize}
	\item einfache und intuitive Handhabung im Betrieb	
		
	\item Mensch-Maschinen-Beziehung zufriedenstellend
	
	\item geringer Unterhalt
	
	\end{itemize}

\end{itemize}

Nach folgender Gewichtung werden die Konzepte dabei bewertet:
\begin{table}[H]
	\begin{tabular}{|l|c|}
	\hline 
	\textbf{Kriterium} & \textbf{Gewichtung} \\ 
	\hline 
	Komplexität & 0.25 \\ 
	\hline 
	Risiken & 0.25 \\ 
	\hline 
	Kosten & 0.05 \\ 
	\hline 
	Verfügbarkeit der Komponenten & 0.2 \\ 
	\hline 
	Innovationsgrad & 0.05 \\ 
	\hline 
	Benutzerfreundlichkeit & 0.2 \\ 
	\hline 
	\end{tabular} 
	\caption{Gewichtung der Bewertungskriterien}
	\label{tab:gewichtung}
\end{table}

Die Gewichtung basiert auf folgenden Überlegungen:
\begin{itemize}
	\item Die Umsetzung eines Funktionsmusters, welches im gesetzten Zeitrahmen realisierbar ist und deren Funktionalität vielversprechend (und dadurch risikoarm) erscheint, ist essentiell für den Erfolg der vorliegenden Bachelorarbeit. Daher werden die Kriterien Komplexität und Risiken am höchsten gewichtet.
	
	\item Weiter ist Verfügbarkeit der Komponenten ein zentrales Thema. Wird ein Konzept ausgewählt, welches Komponenten beinhaltet die nicht lieferbar oder nur verspätet lieferbar sind, droht die Umsetzung zu scheitern. Daher wird dieses Kriterium mit 20\% gewichtet.
	
	\item Das Einsatzgebiet des Planting Robots befindet sich im Garten- und Zierpflanzenbau. Es ist anzunehmen, dass Operatoren geringe technische Kenntnisse besitzen und daher eher abgeneigte Haltung einnehmen, ein solches Gerät zu bedienen. Umso zentraler ist daher eine benutzerfreundliche Interaktion zwischen Mensch und Maschine. Deshalb wird auch dieses Kriterium mit 20\% gewichtet.
	
	\item Das Kriterium Innovationsgrad beeinflusst das Erfüllen der gesetzten Ziele nur indirekt, weshalb dieses mit 5\% gewichtet wird. 
	
	\item Auch die Kosten sind mit 5\% gewichtet. Dies basiert auf der Tatsache, dass alle Ausgaben im Voraus genehmigt werden müssen. Dadurch muss zwingend eine einvernehmliche Lösung mit dem Industriepartner gefunden werden.
\end{itemize}

\subsubsection{Ermittlung des Nutzwertes}
\label{nutzwert}
Nachdem die Kriterien sowie deren Gewichtung definiert sind, folgt die Ermittlung des Nutzwertes. Die objektive Beurteilung der drei Konzepte ergaben folgende Gesamtnoten:
\begin{table}[H]
\begin{tabular}{|L{4cm}|C{2.2cm}|C{2.2cm}|C{2.2cm}|C{2.3cm}|}
	\hline 
	\textbf{Kriterien} & \textbf{Konzept Grau} & \textbf{Konzept Grün} & \textbf{Konzept Blau} & \textbf{Gewichtung} \\ 
	\hline 
	Komplexität & 2 & 3 & 4 & 0.25 \\ 
	\hline 
	Risiken & 1 & 3 & 3 & 0.25 \\ 
	\hline 
	Kosten & 2 & 2 & 4 & 0.05 \\ 
	\hline 
	Verfügbarkeit Komponenten & 4 & 2 & 4 & 0.2 \\ 
	\hline 
	Innovationsgrad & 4 & 3 & 3 & 0.05 \\ 
	\hline 
	Benutzerfreundlichkeit & 4 & 4 & 3 & 0.2 \\ 
	\hline 
	\textbf{Gesamtnote} & \textbf{2.7} & \textbf{3.0} & \textbf{3.5} &  \\ 
	\hline 
\end{tabular} 
	\caption{Ermittlung des Nutzwertes}
	\label{tab:nutzwert}
\end{table}

Für die bessere Nachvollziehbarkeit der einzelnen Punkte wird auf jedes Kriterium eingegangen und die Beurteilung begründet.
\newline

\textbf{Komplexität:}
\begin{itemize}
	\item Für die Realisierung mit Pneumatik (Konzept Grau) fehlt es an technischem Know-how, wodurch eine nur ausreichende Bewertung resultiert. 
	
	\item Die Vereinzelung sowie Förderung von Konzept Grün ist durch den Wendelförderer auf einfache Art realisierbar. Dies ergibt eine gute Beurteilung.
	
	\item Die getestete Vereinzelung sowie Nutzung der Schwerkraft zum Transport der NemaCaps von Konzept Blau, ergibt eine vielversprechende Lösung.
\end{itemize}

\textbf{Risiken:}
\begin{itemize}
	\item Durch die ungewisse Förderung von NemaCaps mittels Pneumatik und dem fehlenden technischen Know-How, wird Konzept Grau als \textit{gerade noch tragbar} bewertet.
	
	\item Aufgrund der gewonnenen Erkenntnisee der Funktionsnachweise erscheinen die Konzepte Grün und Blau als machbar (gut). 
\end{itemize}

\textbf{Kosten:}
\begin{itemize}
	\item Pneumatikkomponenten sind teuer. Weiter fehlen an der Grundausstattung der Topfmaschine Pneumatikanschlüsse, was die Kosten von Konzept Grau weiter erhöht.
	
	\item preiswerte Wendelförderer sind nur über den Import aus China erhältlich. Zudem sind die Anforderungen an die Aktoren von Konzept Grün hoch, was die Beschaffung teureren Komponenten bedeutet.  
\end{itemize}

\textbf{Verfügbarkeit Komponenten:}
\begin{itemize}
	\item Pneumatikkomponenten (Konzept Grau) sind etabliert und dadurch in nützlicher Frist lieferbar. Auch sind Alternativprodukte erhältlich. 
		
	\item Wendelförderer werden nur von einer beschränkten Anzahl Hersteller angeboten, wodurch Konzept Grün nur \textit{ausreichend} bewertet wird. Für die alternative Beschaffung aus China beträgt die Lieferfrist mindestens 6 Wochen.
\end{itemize}

\textbf{Innovationsgrad:}
\begin{itemize}
	\item Der Innovationsgrad wird für jedes Konzept als \textit{gut} oder \textit{sehr gut} beurteilt.
	
	\item Der höchste Innovationsgrad wird in Konzept Grau gesehen.
\end{itemize}

\textbf{Benutzerfreundlichkeit:}
\begin{itemize}
	\item Die Benutzerfreundlichkeit wird bei Konzept Grau und Grün als sehr gut befunden, vorallem durch die simplen Lösungen zur Befüllung des Roboters. Beide Lösungen verfügen über einen Trichter oder Behälter, wodurch die Befüllung benutzerfreundlich gestaltet ist.
	
	\item Konzept Blau verfügt auch über ein Behälter zur Befüllung. Die Bewertung dieses Konzepts fällt jedoch schlechter aus, da sich dieser Behälter in erhöhter Position über der Setzeinheit befindet.
\end{itemize}

\subsubsection{Entscheid}
Aufgrund der ausgeführten Nutzwertanalyse (vgl. Kap. \ref{nutzwert}) wird entschieden, welches Konzept während der vorliegenden Bachelorarbeit realisiert wird. Umgesetzt wird das Konzept mit der besten Gesamtnote. Somit wird in der Umsetzungsphase \textbf{Konzept Blau} realisiert (vgl. Tab. \ref{tab:nutzwert}).