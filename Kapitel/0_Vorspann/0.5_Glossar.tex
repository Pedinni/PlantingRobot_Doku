\newpage
\begin{sffamily}
\phantomsection\addcontentsline{toc}{section}{Glossar} 
\section*{Abkürzungsverzeichnis}\label{dok:glossar}
\begin{table}[H]
	\begin{tabular}{|L{2.5cm}|L{13cm}|}
		\hline	
		\textbf{DC} & Direct Current\\
		
		\hline
		\textbf{GND} & Ground\\
		
		\hline
		\textbf{HSLU} & Hochschule Luzern \\ 
		
		\hline
		\textbf{KDS} & Kinetis Design Studio\\

		\hline
		\textbf{LED} & Light-emitting diode\\
		
	 	\hline
	 	\textbf{PAIND} &  Industrieprojekt (Modul der HSLU) \\ 
	 	
	 	\hline
	 	\textbf{PCB} &	Printed Circuit Board\\
				
		\hline
		\textbf{RTOS} & Real Time Operating System \\
		
		\hline
		\textbf{SoC} &	System on Chip\\
		
		\hline
		\textbf{SWD} &	Serial Wire Debug\\
		
		\hline
		\textbf{UART} &	Universal Asynchronous Receiver Transmitter\\
		
		\hline
		\textbf{USB} &	Universal Serial Bus\\		
		
		\hline
		\textbf{openSDA} & Ein Seriell- und Debugadapter, welcher in diverse NXP Entwicklungsplatformen integriert ist.\\
		
		\hline
		\textbf{HMI} & Human Machine Interface\\
				
		\hline
	\end{tabular} 
	\vspace{0.2cm}
\end{table}


\subsection*{Begrifflichkeiten}
\begin{table}[H]
	
	\begin{tabular}{|L{4.5cm}|L{11cm}|}
		\hline
		\textbf{BLE} & Bluetooth Low Energy: Ein unter Bluetooth 4.0 eingeführter Standard zur Datenübertragung für portable Geräte.\\
		
		\hline		
		\textbf{Demonstrator PCB} & Eine im Umfang dieses Projekts entwickelte Printplatine.\\
		
		\hline		
		\textbf{Dockingstation PCB} & Eine im Umfang dieses Projekts entwickelte Printplatine.\\
		
		\hline		
		\textbf{FRDM-Board} & In diesem Projekt ist damit spezifisch das Freedom Board KL25Z gemeint. Ein Mikrocontroller Entwicklungsboard von NXP mit einem ARM Cortex-M0+ Prozessor. \\
		
		\hline
		\textbf{Hexiwear} & Ein Mikrocontroller Entwicklungsboard von NXP mit einem ARM Cortex-M0+ mit BLE (SoC) sowie einem ARM Cortex-M4 Prozessor und diversen Sensoren.\\
		
		\hline
		\textbf{nRF24L01+} & Ein über entsprechende Funkmodule realisiertes Drahtloses Verbindungsprotokoll.\\	
			
		\hline
		\textbf{Python} & Interpretierte Programmiersprache \\
		
		\hline		
		\textbf{Raspberry} & Raspberry Pi 3 Model B: Ein kompaktes Linux basiertes Computersystem mit einem Broadcom Quad Core Prozessor, 1GB RAM und Peripherie Schnittstellen. \\	
		
		\hline
		\textbf{RGB} &  Farbraum, welcher durch das mischen von rot, blau und grün eine Farbwahrnehmung nachbildet. \\
		
		\hline
		\textbf{NemaCaps} &  Das Setzgut des Planting Robots. Von der Firma MCC Laboratoire Meiners hergestellte Kapseln in Kugelform mit Nematoden als Inhalt. \\
		
		\hline
	\end{tabular} 
	\vspace{0.2cm}
\end{table}


\end{sffamily}