\newpage
\section*{Zusammenfassung}
Anhand einer Funktionsanalyse wurde die Aufgabestellung in verschiedene Funktionen unterteilt, welche den Disziplinen Allgemein, Maschinentechnik, Elektrotechnik und Informatik eingeteilt wurden. Für jede Funktion wurden unterschiedliche Ansätze und mögliche Technologien recherchiert und verglichen.
Die Umsetzungsmöglichkeiten flossen in einen morphologischen Kasten, in welchem die gewählten Gesamtlösungen mit Linien gekennzeichnet wurden.\newline

Nachdem das Entsorgungsfahrzeug eingeschaltet und der Initialisierungsvorgang abgeschlossen ist, wird die Fahrt durch einen der Startbuttons, blau oder grün, gestartet. Während der Fahrt wird mithilfe einer Kamera sichergestellt, dass das Fahrzeug der Spur folgt. Sobald die zweite Kamera einen Abfallcontainer der entsprechenden Farbe erfasst, erkennt die Software diesen und lässt das Fahrzeug nach einer Prüfung mit dem Distanzsensor an der richtigen Stelle halten. Der seitlich fixierte Greifarm wird horizontal ausgefahren und greift nach dem Abfallcontainer, wonach er über eine Rotationsbewegung in den Laderaum entleert wird. Sobald der Abfallcontainer geleert und zurückgestellt ist, fährt das Fahrzeug weiter. Wenn die Software mithilfe der Fahrtkamera eine Kreuzung erkennt, lässt sie mittels Ultraschallsensor den Rechtsvortritt überprüfen. Gelangt das Fahrzeug zur Ziellinie, leitet die Software den Endvorgang ein. Dabei hält das Fahrzeug in der Zielfeld neben dem Entsorgungsbecken in welches der Greifarm den Laderaum kippt.\newline