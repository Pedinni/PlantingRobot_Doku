\newpage
\section{Abstract}
\textit{(ygu/pro)} Der Einsatz von Fadenwürmern zur Schädlingsbekämpfung in der Landwirtschaft sowie im Gartenbau bietet eine biologische Alternative zu konventionellen Pestiziden. Die industrielle Bestückung von Töpfen mit Kapseln, welche Fadenwürmer beinhalten, birgt einen hohen personellen Aufwand im Gartenbau. Dieser Nachteil soll mit dem Einsatz eines Roboters eliminiert werden. Entwickelt wird ein Funktionsmuster eines Roboters, der die automatische Dosierung sowie Platzierung von Nematoden-Kapseln in Töpfen für den Gemüse- und Zierpflanzenbau ausführt. Das Funktionsmuster ist an einer halbautomatischen Topfmaschine (für die Befüllung von Töpfen?) gekoppelt.
\newline

Für die Entwicklung des Funktionsmusters wurde mittels Funktionsanalyse die Aufgabe funktional abstrahiert und in lösungsneutrale Teilfunktionen zerlegt. Die Ausarbeitung von drei Lösungskonzepten erfolgte durch die funktionsbezogene Variation von Teillösungen und einer Konzeptauswahl mittels morphologischer Kasten. Für eine verbesserte Einschätzung der Machbarkeit wurden kritische Teilfunktionen einem praktischen Funktionsnachweis unterzogen. Als Resultat einer Nutzwertanalyse wurde eines der drei ausgearbeiteten Konzepte für die Umsetzung ausgewählt. 
\newline

Während der praktischen Umsetzung des Konzepts erfolgte die Entwicklung der Konstruktion, Software sowie der PCB’s in vermehrt fachspezifischer Arbeit. Die realisierten Komponenten wurden einzelner Funktionstest unterzogen und abschliessend als Gesamtsystem getestet.
\newline

Am Funktionsmuster wurde gezeigt, dass eine industrielle Bestückung von NemaCaps realisierbar ist. Die Vereinzelung sowie der Transport der Kapseln wurde erfolgreich nachgewiesen, wobei die Gesamtfunktion am Ausheben eines Setzloches scheiterte.
\newline

Die gewonnenen Erkenntnisse aus den Funktionstests lassen vermuten, dass durch gezielte Anpassungen an der Setzeinheit die industrielle Bestückung von NemaCaps mit dem gewählten Konzept möglich ist. Für die Erreichung dessen werden konkrete Massnahmen genannt.

\newpage
The application of nematodes for pest control in agriculture and gardening offers a biological alternative to conventional pesticides. The industrial assembly (besser mounting?) of pots with capsules which contain nematodes imply higher human Resources in gardening than usual. This disadvantage is to be eliminated by the application of a robot. The development of a functional prototype of such a robot is to be undertaken which doses and places nematode-capsules in pots automatically. The pots are used for vegetable gardening and ornamental plants. The functional prototype is linked (besser coupled?) with a semi-automatic potting machine.
\newline

The development of this functional prototype was abstracted by a functional analysis. Followed by objective (besser conclusion-neutral?) observation (besser contemplation?), the main-function was split into several sub-functions. Based on the functional analysis, an elaboration of three concepts was done by using creativity techniques and a morphological box. In order to support the assessment of those concepts, critical functions were tested. As an result of an utility analysis, one of the concept was chosen to be realised as functional prototype.
\newline

During the practial implementation of the chosen concept such as the design of the printed curcuit board, the software and the whole mechanical setup was carried out in mainly subject-specific tasks. The assembled components were undertaken functional tests before the system as a whole was undergone an extensive testing procedure.
\newline

At the assembled functional prototype it was shown that an industrial assembly of NemaCaps is basically possible. The separation and transport of NemaCaps was tested successfully, whereas main-function failed due to the fail of preparing a hole in the pot.
\newline

The gained findings from testing the functional prototype lead to the supposition that an industrial assembly of NemaCaps with the chosen concept is likely by changing specific parameter. In order to fulfill that conrete measures are listed.