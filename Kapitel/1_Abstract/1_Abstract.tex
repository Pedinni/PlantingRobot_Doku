\newpage
\section{Abstract}
\textit{(ygu/pro)}Der Einsatz von Fadenwürmern zur Schädlingsbekämpfung in der Landwirtschaft, sowie im Gartenbau bietet eine biologische Alternative zu konventionellen Pestiziden. Die industrielle Bestückung von Töpfen mit Kapseln, welche Fadenwürmer beinhalten, birgt im Gartenbau einen hohen personellen Aufwand. Dieser Nachteil soll durch den Einsatz eines Pflanzroboters beseitigt werden. Entwickelt wird ein Funktionsmuster eines Roboters, welches die automatische Dosierung, wie auch die Platzierung von Nematoden-Kapseln in Töpfen für den Gemüse- und Zierpflanzenbau ausführt. Der Roboter ist an eine halbautomatische Topfmaschine gekoppelt.
\newline

Für die Entwicklung des Funktionsmusters wurde mittels Funktionsanalyse die Aufgabe funktional abstrahiert und in lösungsneutrale Teilfunktionen zerlegt. Die Ausarbeitung von drei Lösungskonzepten erfolgte durch die funktionsbezogene Variation von Teillösungen und einer Konzeptauswahl mittels morphologischer Kasten. Für eine verbesserte Einschätzung der Machbarkeit wurden kritische Teilfunktionen einem praktischen Funktionsnachweis unterzogen. Als Resultat der Nutzwertanalyse wurde eines der drei Konzepte für die Umsetzung ausgewählt.
\newline

Während der praktischen Umsetzung des Konzepts erfolgte die Entwicklung der Konstruktion, Software sowie der PCB’s in vermehrt fachspezifischer Arbeit. Die erarbeiteten Komponenten wurden durch Funktionstests geprüft und zu letzt als Gesamtsystem getestet.
\newline

Anhand des Funktionsmusters konnte aufgezeigt werden, dass eine industrielle Bestückung von NemaCaps für den alltäglichen Gebrach umsetzbar/realisierbar ist.
Die Vereinzelung und der Transport der Kapseln konnte durch den Roboter erfüllt werden, die Gesamtfunktion des Funktionsmusters scheiterte jedoch am Ausheben eines Setzlochs.
\newline

Die gewonnenen Erkenntnisse aus den Funktionstests lassen vermuten, dass durch gezielte Anpassungen an der Setzeinheit die industrielle Bestückung von NemaCaps mit dem gewählten Konzept möglich ist. Für die Erreichung dessen werden konkrete Massnahmen genannt.

\newpage
The application of nematodes for pest control in agriculture and gardening offers a biological alternative to conventional pesticides. The industrial assemby of pots with capsules which contain nematodes implies higher human Resources in gardening than usual. This disadvantage is to be removed by the application of a robot. The development of a functional prototype of such a robot is to be undertaken which doses and places nematode-capsules in pots automatically. The pots are used for vegetable gardening and ornamental plants. The functional prototype is linked with a semi-automatic potting machine.
\newline

The development of this functional prototype was abstracted by a functional analysis. Furthermore an objective observation was undertaken and the main-function of the prototype was split into several sub-functions. Based on the functional analysis, an elaboration of three concepts was done by using creativity techniques and a morphological box. In order to support the assessment of those concepts, critical functions were tested. One of those concepts was chosen to be realised as a consequence of the utility analysis.
\newline

During the practial implementation of the chosen concept such as the design of the printed curcuit board, the software and the whole mechanical setup was carried out in mainly subject-specific tasks. The assembled components were verified by several functional examinations before the system as whole was undergone an extensive testing procedure.
\newline

As Conclusion the assembled functional prototype was shown to be able of carrying NemaCaps. The separation and transport of NemaCaps was tested successfully, whereas main-function failed due to the fail of preparing a hole in the pot.
\newline

The gained findings from testing the functional prototype lead to the supposition that an industrial assembly of NemaCaps with the chosen concept is likely by changing specific parameter. In order to fulfill that conrete measures are listed.