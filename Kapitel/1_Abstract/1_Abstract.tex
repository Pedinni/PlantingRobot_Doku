\newpage
\section{Abstract}
Der Einsatz von Fadenwürmern zur Schädlingsbekämpfung in der Landwirtschaft sowie im Gartenbau bietet eine biologische Alternative zu konventionellen Pestiziden. Die industrielle Bestückung von Töpfen mit Kapseln, welche Fadenwürmer beinhalten, birgt einen hohen personellen Aufwand im Gartenbau. Dieser Nachteil soll mit dem Einsatz eines Roboters eliminiert werden. Entwickelt wird ein Funktionsmuster eines Roboters, der die automatische Dosierung sowie Platzierung von Nematoden-Kapseln in Töpfen für den Gemüse- und Zierpflanzenbau ausführt. Das Funktionsmuster ist an einer halbautomatischen Topfmaschine (für die Befüllung von Töpfen?) gekoppelt.
\newline

Für die Entwicklung des Funktionsmusters wurde mittels Funktionsanalyse die Aufgabe funktional abstrahiert und in lösungsneutrale Teilfunktionen zerlegt. Die Ausarbeitung von drei Lösungskonzepten erfolgte durch die funktionsbezogene Variation von Teillösungen und einer Konzeptauswahl mittels morphologischer Kasten. Für eine verbesserte Einschätzung der Machbarkeit wurden kritische Teilfunktionen einem praktischen Funktionsnachweis unterzogen. Als Resultat einer Nutzwertanalyse wurde eines der drei ausgearbeiteten Konzepte für die Umsetzung ausgewählt. 
\newline

Während der praktischen Umsetzung des Konzepts erfolgte die Entwicklung der Konstruktion, Software sowie der PCB’s in vermehrt fachspezifischer Arbeit. Die realisierten Komponenten wurden einzelner Funktionstest unterzogen und abschliessend als Gesamtsystem getestet.
\newline

Teil Resultate
\newline
Teil Diskussion
