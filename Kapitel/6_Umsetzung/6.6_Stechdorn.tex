\subsection{Stechdorn}
Der Stechdorn hat die Aufgabe, ein Setzloch durch Verdrängung der Topferde auszuheben und anschliessend die fallenden NemaCaps ins Setzloch zu leiten. Dafür wurde auf einen gelochten Stechdorn bewusst verzichtet, da man das Risiko von Verstopfungen nicht eingehen möchte. Mit der realisierten Konstruktion wird dieses Risiko eliminiert. 
\newline
\textbf{Aufbau}
\newline

Der Stechdorn besteht aus mehreren Teilen, wobei diese über eine lineare Führung (Punkt 5 in Abbildung \ref{fig:details_stechdorn}) verbunden sind. Dabei wird das Haupt (1) an der Verstellmechanik montiert und macht so die Translation der Setzeinheit mit. Sobald der untere Teil (2, 3) in die Topferde einsticht, fährt die Spitze an den oberen Anschlag der Führung (Detail A). Bei der Bewegung zurück nach oben öffnet sich die Spitze wieder und das NemaCaps kann durch den Kanal (8) ins ausgehobene Setzloch fallen (Detail B). Dabei soll die Bewegung nur durch die Gewichts- sowie Trägheitskraft der Spitze ausgelöst werden. Falls die Spitze zu leicht ist, um sich durch die Bewegung zu öffnen, kann in der Spitze Blei im vorgesehenen Hohlraum eingefüllt werden. Dafür ist ein Füllloch (4) vorgesehen, welches mit einer Madenschraube M6 verschlossen wird.
\newline
Folgende Überlegungen geben die Lage der linearen Führung vor:
\begin{itemize}
	\item Die lineare Führung soll möglichst parallel zur Bewegungsachse der Setzeinheit verlaufen, sodass möglichst keine Radialkräfte auf den Dorn wirken.
	
	\item Wiederum muss eine seitliche Öffnung der Spitze soweit geschehen, dass die Öffnung (9) frei über dem Setzloch steht (Detail B) und ein freier Fall des NemaCaps möglich ist.
	
	\item Die Verschliessung der Spitze ist in vertikaler Richtung beschränkt durch die Auslegung Spindel. Dabei ist der Abstand A in den Berechnungen (Siehe Anhang \textbf{XY}) mit 15mm angenommen. Die Umsetzung überschreitet diese Annahme um 1mm, bewegt jedoch im angenommenen Rahmen.
\end{itemize}

	\begin{figure}[H]
	\includegraphics[scale=1.0]{Illustrationen/6-Umsetzung/details_stechdorn.jpg}
	\caption{Übersicht des Stechdorns}
	\label{fig:details_stechdorn}
	\end{figure}
\textbf{Haupt}
\newline
Das Haupt des Stechdorns nimmt neben der linearen Führung der Spitze noch folgende Aufgaben wahr:
\begin{itemize}
	\item Es verbindet den Schlauch mit dem Stechdorn. Dafür kann der Schlauch oben eingeführt werden und am Schlauchhalter (Punkt 12 in Abbildung \ref{fig:details_haupt}) mit einem Kabelbinder montiert werden.
	
	\item Es lenkt das fallende NemaCap mit dem Kanal (8) zur vorgesehen Öffnung für die Platzierung. In der Schnittansicht A -A ist zu sehen, dass der Kanal kurz vor Austritt einen kegelförmigen Verlauf hat. 
	
	\item Mit der Anbringung eines Stiftes (13) wird der untere Anschlag der Führung gewährleistet. 
		
	\item Die Montage des Dorns an der Verstellmechanik.
\end{itemize}

	\begin{figure}[H]
	\includegraphics[scale=0.62]{Illustrationen/6-Umsetzung/details_haupt.PNG}
	\caption{Details zum Haupt}
	\label{fig:details_haupt}
	\end{figure}

\textbf{Spitze oben}
\newline
An der Spitze oben sind folgende konstruktive Überlegungen hervor zu heben:
\begin{itemize}
	\item Die lineare Führung ist als breites T-Profil umgesetzt (Detail A in Abbildung \ref{fig:details_spitze_oben}). Dabei ist der Steg 1.8mm dick (Mass E). Als Spielmass zwischen beiden Führungsteilen wird in alle Richtungen 0.25mm verwendet.
	
	\item Bei ausgefahrener Spitze beträgt die verbleibende Länge der Führung 11mm (Mass D). Gemäss betreuendem Dozenten (Marco De Angelis) ist das Verhältnis p von Stegdicke zu verbleibender Länge essentiel bei der Realisation von linearen Führungen dieser Art. Ideal ist ein Verhältnis von 8 ... 10 anzustreben. in diesem Fall beträgt das Verhältnis p:
	
	\begin{equation}
	p=\frac{verbleibende Fuehrung}{Stegdicke}=\frac{D}{E}=\frac{11mm}{1.8mm}=6.11
	\end{equation}
		
	Ob ein Verhältnis von 6.1 für eine funktionierende Führung ausreicht, wird die Inbetriebnahme zeigen. Allenfalls müssen diese Parameter angepasst werden.
	
	\item Da die Spitze direkt an der Topferde ausgesetzt ist, muss mit Kontakt von Erdpartikel gerechnet werden. Dabei ist die Führung ein sensitiver Teil, deren Funktion bei Kontakt mit Erdpartikeln beeinträchtigt werden kann. Hierfür wurde vorbeugend Massnahmen in der Konstruktion berücksichtigt, um dieses Risiko zu mindern. Ein Auslauf der Führung (11) und zwei Fasen (10) sollen die Sammlung von Erdpartikeln verhindern.
	\end{itemize}
	\begin{figure}[H]
	\includegraphics[scale=1.0]{Illustrationen/6-Umsetzung/details_spitze_oben.jpg}
	\caption{Details Spitze oben}
	\label{fig:details_spitze_oben}
	\end{figure}

\textbf{Materialwahl und Fertigungsverfahren}
\newline
Die komplexe Geometrie dieser Teile sind ausschlaggebend, dass Rapid Prototyping als Fertigungsverfahren gewählt wird. Bewusst wird die Interalbauweise für den Stechdorn genutzt, um viele Funktionen in einem Teil zu vereinen. Durch erhöhten Anforderungen an das Gewicht wird ein Kunststoff (ABS) als Material verwendet. Dabei ist es möglich, dass ein Teil mehrmals gedruckt wird, da das optimale Spielmass der linearen Führung iterativ durch zielgerichtetes Ausprobieren gefunden werden muss. 
\newline
\textbf{Schlauch}
\newline
Für den Transport der NemaCaps zwischen Vereinzelung und Stechdorn wird die Gravitation verwendet. Dabei werden die fallenden NemaCaps durch einen Pneumatikschlauch geleitet. Auch stellt der Schlauch sicher, dass eine flexible Verbindung zum bewegten Stechdorn existiert. In der Pneumatik sind folgende Schlauchdimensionen gängig:
\newline
\begin{table}[H]
\begin{tabular}{|c|c|}
	\hline 
	Aussendurchmesser D [mm] & Innendurchmesser d [mm] \\ 
	\hline 
	5 & 3 \\ 
	\hline 
	6 & 4  \\ 
	\hline 
	8 & 6 \\ 
	\hline 
	10 & 8  \\ 
	\hline 
\end{tabular}
	\caption{gängige Schlauchdimensioen}
	\label{tab:Schlauchdimensioen}
\end{table}

Für die Wahl der Schlauchdimensionen sind folgende Überlegungen entscheidend:
\begin{itemize}
	\item NemaCaps haben einen Durchmesser bis zu 3.6mm. Für einen freien Fall soll ein möglichst grosser Freiraum zur Schlauchinnenwand herrschen.
	
	\item Ein grosser Aussendurchmesser des Schlauches hat direkten Einfluss auf die Schlauchkupplungen sowie die Verstellmechanik. Dadurch werden diese Komponenten grösser. Der Lochabstand der Lochmaske steigt an sowie die Verstellmechanik muss grösser dimensioniert werden. Dies bedeutet mehr translatorisch beschleunigte Masse durch die Spindel. Somit ist hierfür die Wahl eines kleinen Schlauchdurchmessers vorteilhaft.
\end{itemize}
Dieser Zielkonflikt beider Überlegungen bedeutet, dass die optimale Wahl des Schlauches ein Kompromiss sein muss. Daher wird ein Schlauch mit den Dimensionen 8/6 (D/d) verwendet.
\newline
Die Flexibilität wird durch das Material massgebend beeinflusst. Da keine Herstellerangaben dafür verfügbar sind, werden die Materialien Polyurethan sowie Polyamid (Nylon) für die Inbetriebnahme bestellt und dort miteinander verglichen.

\textbf{Elektrostatische Aufladung behandeln?}