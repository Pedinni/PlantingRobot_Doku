\subsection{Mainboard PCB}
\subsubsection{FRDM-KL25Z Header}
Das Mainboard PCB verbindet sämtliche Elektronik Komponenten miteinander. Das Herzstück bildet dabei das Mikrocontroller Board FRDM-KL25Z von NXP, welches über Stiftleisten auf das Mainboard aufgesteckt wird. Es verfügt über einen 32-bit ARM Cortex M0+ Mikrocontroller mit einer Taktrate von 48MHz, 128KB Flash und 16KB RAM. In der folgenden Tabelle werden die Signale welche vom FRDM-Board zur Verfügung gestellt werden aufgeführt. Die Spalten der Tabelle sind wie folgt zu lesen: \textbf{CON} ist der Stecker mit der jeweiligen Pin Nummer, \textbf{Port} beschreibt den Pin des uC, \textbf{Mapping} gibt an für was der Port des uC verwendet wird und \textbf{Funktion} beschreibt die Verwendung des uC Pins auf dem Mainboard PCB.

% Table generated by Excel2LaTeX from sheet 'FRDM-KL25z Port Mapping'
\begin{table}[H]
	\scriptsize
	\centering
	\caption{FRDM-KL25z Port Mapping}
	\begin{tabular}{|r|r|r|r|l|l|r|l|}
		\hline
		\multicolumn{1}{|l|}{\textbf{CON}} & \multicolumn{1}{l|}{\textbf{Port}} & \multicolumn{1}{l|}{\textbf{Mapping}} & \multicolumn{1}{l|}{\textbf{Funktion}} & \textbf{CON} & \textbf{Port} & \multicolumn{1}{l|}{\textbf{Mapping}} & \textbf{Funktion} \\
		\hline
		\multicolumn{1}{|l|}{J1 01} & \multicolumn{1}{l|}{PTC7} &       & \multicolumn{1}{l|}{NC} & J2 01 & PTC12 & \multicolumn{1}{l|}{Digital In} & BTN\_tiefer \\
		\hline
		\multicolumn{1}{|l|}{J1 02} & \multicolumn{1}{l|}{PTA1 } & \multicolumn{1}{l|}{UART0\_RX} & \multicolumn{1}{l|}{Bluetooth RX} & J2 02 & PTA13  & \multicolumn{1}{l|}{FTM1\_CH1} & Vibramotors 1 \\
		\hline
		\multicolumn{1}{|l|}{J1 03} & \multicolumn{1}{l|}{PTC0} & \multicolumn{1}{l|}{ADC0\_SE14} & \multicolumn{1}{l|}{IR-Sensor} & J2 03 & PTC13 & \multicolumn{1}{l|}{Digital In} & BTN\_höher \\
		\hline
		\multicolumn{1}{|l|}{J1 04} & \multicolumn{1}{l|}{PTA2 } & \multicolumn{1}{l|}{UART0\_TX} & \multicolumn{1}{l|}{Bluetooth TX} & J2 04 & PTD5  & \multicolumn{1}{l|}{Digital Out} & CSN nRF24L01+ \\
		\hline
		\multicolumn{1}{|l|}{J1 05} & \multicolumn{1}{l|}{PTC3} &       & \multicolumn{1}{l|}{NC} & J2 05 & PTC16 & \multicolumn{1}{l|}{Digital In} & BTN Vereinzeln \\
		\hline
		\multicolumn{1}{|l|}{J1 06} & \multicolumn{1}{l|}{PTD4 } & \multicolumn{1}{l|}{Digital Out} & \multicolumn{1}{l|}{DC\_OK} & J2 06 & PTD0  & \multicolumn{1}{l|}{Digital Out} & CE nRF24L01+ \\
		\hline
		\multicolumn{1}{|l|}{J1 07} & \multicolumn{1}{l|}{PTC4} &       & \multicolumn{1}{l|}{NC} & J2 07 & PTC17 & \multicolumn{1}{l|}{Digital In} & BTN Spindel runter \\
		\hline
		\multicolumn{1}{|l|}{J1 08} & \multicolumn{1}{l|}{PTA12 } & \multicolumn{1}{l|}{FTM1\_CH0} & \multicolumn{1}{l|}{Vibramotors 2} & J2 08 & PTD2  & \multicolumn{1}{l|}{SPI0\_MOSI} & nRF24L01+ \\
		\hline
		\multicolumn{1}{|l|}{J1 09} & \multicolumn{1}{l|}{PTC5} &       & \multicolumn{1}{l|}{NC} & J2 09 & PTA16  & \multicolumn{1}{l|}{Digital In} & BTN Spindel hoch \\
		\hline
		\multicolumn{1}{|l|}{J1 10} & \multicolumn{1}{l|}{PTA4 } & \multicolumn{1}{l|}{Digital Out} & \multicolumn{1}{l|}{Servo enable} & J2 10 & PTD3  & \multicolumn{1}{l|}{SPI0\_MISO} & nRF24L01+ \\
		\hline
		\multicolumn{1}{|l|}{J1 11} & \multicolumn{1}{l|}{PTC6} &       & \multicolumn{1}{l|}{NC} & J2 11 & PTA17  &       & NC \\
		\hline
		\multicolumn{1}{|l|}{J1 12} & \multicolumn{1}{l|}{PTA5 } & \multicolumn{1}{l|}{FTM0\_CH2} & \multicolumn{1}{l|}{ION enable} & J2 12 & PTD1  & \multicolumn{1}{l|}{SPI0\_SCK} & nRF24L01+ \\
		\hline
		\multicolumn{1}{|l|}{J1 13} & \multicolumn{1}{l|}{PTC10} & \multicolumn{1}{l|}{I2C1\_SDA} & \multicolumn{1}{l|}{LED Driver} & J2 13 & PTE31 & \multicolumn{1}{l|}{FTM0\_CH4} & Servo 3 \\
		\hline
		\multicolumn{1}{|l|}{J1 14} & \multicolumn{1}{l|}{PTC8 } & \multicolumn{1}{l|}{Digital Out} & \multicolumn{1}{l|}{Stop\_In} & J2 17 & PTD6  &       & NC \\
		\hline
		\multicolumn{1}{|l|}{J1 15} & \multicolumn{1}{l|}{PTC11} & \multicolumn{1}{l|}{I2C1\_SCL } & \multicolumn{1}{l|}{LED Driver} & J2 18 & PTE1  & \multicolumn{1}{l|}{UART1\_RX} & TMCM-1630 TX \\
		\hline
		\multicolumn{1}{|l|}{J1 16} & \multicolumn{1}{l|}{PTC9 } & \multicolumn{1}{l|}{Digital Out} & \multicolumn{1}{l|}{IRQ nRF24L01+} & J2 19 & PTD7  &       & NC \\
		\hline
		\multicolumn{1}{|l|}{J10 01} & \multicolumn{1}{l|}{PTE20} & \multicolumn{1}{l|}{ADC0\_SE0} & \multicolumn{1}{l|}{IR Sensor 2} & J2 20 & PTE0  & \multicolumn{1}{l|}{UART1\_TX } & TMCM-1630 RX \\
		\hline
		\multicolumn{1}{|l|}{J10 02} & \multicolumn{1}{l|}{PTB0 } & \multicolumn{1}{l|}{ADC0\_SE8} & \multicolumn{1}{l|}{IR Sensor 1} & J9 01 & PTB8  & \multicolumn{1}{l|}{Digital In} & BTN\_AUTO \\
		\hline
		\multicolumn{1}{|l|}{J10 03} & \multicolumn{1}{l|}{PTE21} &       & \multicolumn{1}{l|}{NC} & J9 03 & PTB9  & \multicolumn{1}{l|}{Digital In} & BTN\_14cm \\
		\hline
		\multicolumn{1}{|l|}{J10 04} & \multicolumn{1}{l|}{PTB1 } &       & \multicolumn{1}{l|}{Spare GPIO} & J9 05 & PTB10 & \multicolumn{1}{l|}{Digital In} & BTN\_13cm \\
		\hline
		\multicolumn{1}{|l|}{J10 05} & \multicolumn{1}{l|}{PTE22 } & \multicolumn{1}{l|}{UART2\_TX} & \multicolumn{1}{l|}{RoboClaw RX} & J9 07 & PTB11 & \multicolumn{1}{l|}{Digital In} & BTN\_12cm \\
		\hline
		\multicolumn{1}{|l|}{J10 06} & \multicolumn{1}{l|}{PTB2 } &       & \multicolumn{1}{l|}{Spare GPIO} & J9 09 & PTE2  & \multicolumn{1}{l|}{Digital In} & BTN\_11cm \\
		\hline
		\multicolumn{1}{|l|}{J10 07} & \multicolumn{1}{l|}{PTE23} & \multicolumn{1}{l|}{UART2\_RX} & \multicolumn{1}{l|}{RoboClaw TX} & J9 10 & P5V\_USB &       & NC \\
		\hline
		\multicolumn{1}{|l|}{J10 08} & \multicolumn{1}{l|}{PTB3 } &       & \multicolumn{1}{l|}{Spare GPIO} & J9 11 & PTE3  & \multicolumn{1}{l|}{Digital In} & BTN\_9cm \\
		\hline
		\multicolumn{1}{|l|}{J10 09} & \multicolumn{1}{l|}{PTE29} & \multicolumn{1}{l|}{FTM0\_CH2} & \multicolumn{1}{l|}{Servo 1} & J9 12 & GND   & \multicolumn{1}{l|}{GND} & GND \\
		\hline
		\multicolumn{1}{|l|}{J10 10} & \multicolumn{1}{l|}{PTC2} &       & \multicolumn{1}{l|}{Spare GPIO} & J9 13 & PTE4  &       & NC \\
		\hline
		\multicolumn{1}{|l|}{J10 11} & \multicolumn{1}{l|}{PTE30} & \multicolumn{1}{l|}{FTM0\_CH3} & \multicolumn{1}{l|}{Servo 2} & J9 14 & GND   & \multicolumn{1}{l|}{GND} & GND \\
		\hline
		\multicolumn{1}{|l|}{J10 12} & \multicolumn{1}{l|}{PTC1} &       & \multicolumn{1}{l|}{Spare GPIO} & J9 15 & PTE5  &       & Spare GPIO \\
		\hline
		&       &       &       & J9 16 & P5-9V\_VIN &       & 5VCC \\
		\hline
	\end{tabular}%
	\label{tab:addlabel}%
\end{table}%

Beim Design des Mainboard PCBs wurde darauf geachtet, dass möglichst viele GPIO Pins des FRDM-Boards genutzt werden. Deshalb wurde zusätzlich zu den in Abbildung \ref{fig:Blockschaltbild_Komponenten} aufgeführten Peripherien, ein Servo Interface mit drei Servo Anschlüssen und eine Buchsenleiste mit Reserve GPIOs vorgesehen.

\subsubsection{Power Supply}
Das Mainboard wird über das selbe 12V Netzteil mit Strom versorgt wie der ION Motion Motorcontroller. Um die Stromversorgung des Motorcontrollers schalten zu können wird diese über das Mainboard geführt. Auf dem Mainboard PCB befindet sich dann das Relais U3, welches vom uC gesteuert wird. Ebenfalls schaltbar ist die 5V Speisung des Servo Interfaces über das Relais U2. Dieses Spannungspotential wird vom ION Motion Controller abgegriffen. Dieser Verfügt über einen 5V Spannungswandler welcher bis zu 3A liefern kann.

\begin{figure}[H]
	\includegraphics[width=1\textwidth]{Illustrationen/6-Umsetzung/Schema_Mainboard_PowerSupply.png}
	\caption{Mainboard Power Supply}
	\label{fig:Schema_Mainboard_PowerSupply}
\end{figure}

Die Onboard Spannungsversorgung für das FRDM-Board, den LED Treiber, die Sensoren und Vibrationsmotoren wird durch den DC/DC-Festspannungsregler LT3971 von Linear Technology bereit gestellt. In der folgenden Tabelle sind die wichtigsten Daten des Buck Reglers ausgeführt:

% Table generated by Excel2LaTeX from sheet 'LT3971 Datasheet'
\begin{table}[htbp]
	\small
	\centering
	\caption{LT3971 Datenblatt Auszug}
	\begin{tabular}{|l|ccc|r|}
		\hline
		\textbf{Parameter} & \multicolumn{1}{l}{\textbf{MIN}} & \multicolumn{1}{l}{\textbf{TYP}} & \multicolumn{1}{l}{\textbf{MAX}} & \textbf{Einheit} \\
		\hline
		Eingangsspannung &       &       & 38    & V \\
		\hline
		Ausgangsspannung & 4.93  & 5     & 5.07  & V \\
		\hline
		Ausgangsstrom &       &       & 1.2   & A \\
		\hline
		Schaltfrequenz & 0.2   &       & 2     & MHz \\
		\hline
		Wirkungsgrad &       & 85    &       & \% \\
		\hline
	\end{tabular}%
	\label{tab:LT3971_Datasheet}%
\end{table}%

\begin{figure}[H]
	\includegraphics[width=1\textwidth]{Illustrationen/6-Umsetzung/Mainboard_3D.png}
	\caption{Mainboard PCB}
	\label{fig:Mainboard_3D}
\end{figure}

\begin{figure}[H]
	\includegraphics[width=0.5\textwidth]{Illustrationen/6-Umsetzung/Baseboard_3D_2.png}
	\caption{Baseboard PCB}
	\label{fig:Baseboard_3D}
\end{figure}


