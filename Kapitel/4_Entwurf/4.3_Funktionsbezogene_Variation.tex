\subsection{Funktionsbezogene Variation}
Nach dem in der Funktionsanalyse die Aufgabenstellung in funktioneller Betrachtung konkretisiert ist, folgt die funktionsbezogene Variation. In dieser Phase unzählige Lösungsvarianten für die definierten Teilfunktionen ausgearbeitet \cite{pahl}. Eine Lösungsvariante ist die konkrete technische Umsetzung einer Teilfunktion, welche die Aufgabe der Teilfunktion erfüllt. Dabei wird eine grösstmögliche Anzahl Lösungen für jede Teilfunktion angestrebt. 
\newline
Kreativität und eine unvoreingenommene Betrachtung des Problems ist in dieser Phase von hoher Wichtigkeit. Brainstorming ist eine Kreativitätstechnik die als Hilfsmittel für die Lösungsfindung angewandt wird. Durch die Anwendung solcher Techniken soll das Potenzial dieser kreativen Phase grösstmöglich ausgeschöpft werden.
\newline
Begleitet wird die Ausarbeitung der Teillösungen von einer Technologierecherche. Durch Recherche im Internet und einschlägiger Literatur werden weitere Lösungen hinzugefügt. Auch werden Gefundene Ideen mit bekannten Technologien abgeglichen und verknüpft. Dabei geht es auch darum, sich mit dem aktuellen Stand der Technik zu befassen. 
\newline
So entsteht für jede Teilfunktion eine breite Auswahl an Teillösungen. Diese Auswahl bietet die Grundlage für Ausarbeitung der Lösungskonzepte und somit den weiteren Verlauf der Produktentwicklung. Alle gefundenen Teillösungen sind im Dokument Funktionsbezogene Variation aufgelistet und kurz beschrieben. Auch sind weiterführende Informationen darin vermerkt.
