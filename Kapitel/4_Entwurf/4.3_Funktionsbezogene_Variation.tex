\subsection{Funktionsbezogene Variation}
\label{funktionsbez_var}
\textit{(ygu)} Nachdem mithilfe der Funktionsanalyse die Aufgabenstellung konkretisiert wurde, folgt im Anschluss die funktionsbezogene Variation. In dieser Phase werden unzählige Lösungsansätze für die definierten Teilfunktionen entwickelt \cite{pahl}. Für jede Teilfunktion werden die grösstmögliche Anzahl Lösungen angestrebt.
\newline
Kreativität und eine unvoreingenommene Betrachtung des Problems ist in dieser Phase von grosser Bedeutung. Brainstorming ist eine Kreativitätstechnik die als Hilfsmittel für die Lösungsfindung angewandt wird. Durch die Anwendung solcher Techniken soll das Potenzial der kreativen Phase grösstmöglich ausgeschöpft werden.
\newline
Die Ausarbeitung der Teillösungen wird durch eine Technologierecherche vervollständigt/erweitert. Durch Nachforschungen im Internet und einschlägiger Literatur werden weitere Lösungen hinzugefügt, erarbeitete Ideen mit bekannten Technologien abgeglichen und verknüpft. Wichtig ist eine intensive Auseinandersetzung mit dem aktuellen Stand der Technik.
\newline
Für jede Teilfunktion entsteht folglich eine breite Auswahl an Teillösungen, die als Grundlage für die Ausarbeitung der Lösungskonzepte dient und somit den weiteren Verlauf der Produktentwicklung bestimmt. Alle gefundenen Teillösungen sind im Dokument „Funktionsbezogene Variation“ (vgl. Anhang) aufgelistet. In dieser Aufführung sind nähere Informationen einsehbar.

\textbf{\textit{Funktionsbezogene Variation durch Variation der Teilfunktionen ersetzen?}}