\subsection{Funktionsanalyse}

\textit{(pro)} In diesem Kapitel wird der Planting Robot in verschieden Funktionsblöcke zerlegt. Die dadurch definierten Teilfunktionen ermöglichen ein systematisches Maschinendesign. Zu den jeweiligen Funktionsblöcken werden in Kapitel \ref{funktionsbez_var} verschiedene Lösungsvarianten ausgearbeitet. Diese Teilkonzepte werden anschliessend in Kapitel \ref{konzept} zu einem kompletten Maschinendesign zusammengeführt.\newline
Bei der Funktionsanalyse wird zwischen einer Pflicht- und einer anspruchsvolleren  Wunschanforderung unterschieden. Die Wunschanforderung beschreibt zusätzlich zum vollen Funktionsumfang, eine selbstständig Konfiguration des Planting Robots, die auf die verschiedenen Topfgrössen anwendbar ist.

\begin{figure}[H]
	\includegraphics[width=0.8\textwidth]{Illustrationen/4-Entwurf/Funktionsanalyse_Pflicht.png}
	\caption{Funktionsanalyse Pflicht Blockdiagramm}
	\label{fig:FunktPflicht}
\end{figure}

\begin{itemize}
	\item \textbf{Initialisierung:} Die Initialisierung wird durch einen Operator ausgeführt. Diese Teilfunktion ist nur in der Pflichtanforderung vorhanden, da die Maschine im Umfang der Pflichtanforderung die Initialisierung nicht selbstständig durchführt.
	
	\begin{itemize}
		\item \textbf{Topfgrösse konfigurieren:} In diesem Funktionsblock wird an der Maschine über ein HMI die verwendete Topfgrösse eingestellt.

		\item \textbf{Setzmechanismus konfigurieren:} Der Setzmechanismus muss für verschiedene Topfgrössen eingestellt oder ausgetauscht werden.
	\end{itemize}
	
	\item \textbf{NemaCaps lagern:} Das Setzgut (NemaCaps) wird in der Maschine mit einem Bestand von bis zu 10'000 Einheiten gelagert.
	
	\begin{figure}[H]
		\includegraphics[width=0.8\textwidth]{Illustrationen/4-Entwurf/Funktionsanalyse_Wunsch.png}
		\caption{Funktionsanalyse Wunsch Blockdiagramm}
		\label{fig:FunktWunsch}
	\end{figure}
	
	\item \textbf{NemaCaps fördern:} Diese Teilfunktion behandelt die Verbindung zwischen Lager und Setzmechanismus.
	
	\begin{itemize}
		\item \textbf{NemaCaps vereinzeln:} Um die NemaCaps gezielt und kontrolliert in die Topferde einsetzen zu können, werden diese vor dem Setzen vereinzelt.
		
		\item \textbf{NemaCaps transportieren:} Der Transport zwischen Lager und Setzmechanismus kann vor oder nach der Vereinzelung stattfinden.
	\end{itemize}




	\item \textbf{Topf erkennen:} Diese Teilfunktion übernimmt implizit zwei Aufgaben. Es soll erkannt werden, ob ein Topf für den Setzprozess bereit steht und ob sich dieser in Bewegung ist oder nicht. 
	
	\item \textbf{NemaCaps setzen:} Erst wenn ein Topf bereit steht, wird der Setzprozess eingeleitet. Dieser unterscheidet sich, wie in Abb. \ref{fig:FunktPflicht} und Abb. \ref{fig:FunktWunsch} ersichtlich, zwischen Pflicht und Wunschanforderung anhand des Funktionsumfangs.
	
	\begin{itemize}
		\item \textbf{Topfgrösse erkennen:} Durch Sensorik soll die Topfgrösse jedes Topfes vermessen werden.
		
		\item \textbf{Setzmechanismus konfigurieren:} Anhand der gewonnen Daten zur Topfgrösse, soll die Maschine den Setzmechanismus selbstständig adaptieren.
		
		\item \textbf{Nemacaps platzieren:} Es folgt der eigentliche Setzprozess, in welchem die Nemacaps in einer definierten Anordnung in Position gebracht werden.
		
		\item \textbf{Setzvorgang auslösen:} Die Nemacaps welche vorher in Position gebracht wurden, werden nun in diesem Schritt vom Setzmechanismus in die Erde befördert.
	\end{itemize}
	
\end{itemize}
