\newpage
\section{Schlusswort}

\textbf{Maschinentechnik - Yves Gubelmann}
\newline
\textit{(ygu)} Die Bachelorarbeit begleitete mich durch das sechste Semester im Studium Maschinentechnik an der Hochschule Luzern. Die erhaltene Aufgabenstellung schien mir zu Beginn etwas abstrakt und nur grob definiert. Die offene Formulierung der Aufgabenstellung liess uns einen grossen Gestaltungsfreiraum und konfrontierte uns gleichzeitig mit Eigenverantwortung. Während der Entwurfsphase nahmen wir durch das Festlegen der Rahmenbedingungen eine aktive Rolle ein, was ich sehr schätzte. Der stete Austausch mit dem Industriepartner und den betreuenden Dozenten führte in dieser Phase zu einem klaren Verständnis der Aufgabenstellung sowie Vorgehensweise.
\newline

Die Konzeptausarbeitung empfand ich als spannendsten Prozess während der gesamten Bachelorarbeit. Es entstanden unzählige interessante Lösungsansätze, welche nur im interdisziplinären Austausch mit meinem Arbeitskollegen möglich wurden. Die kreative Findung von innovativen Lösungen war bereichernd, jedoch auch entscheidend für den weiteren Verlauf dieser Arbeit. Nur eine sorgfältige Abwägung aller Kriterien gewährleistete eine pragmatische Wahl des Lösungskonzeptes.
\newline

Die Umsetzungsphase bleibt mir als intensiv und lernreich in Erinnerung. Die Konstruktion des Funktionsmusters stellte mich vor Herausforderungen, die für mich als gelernter Elektroniker unbekannt waren. Vor allem in den Bereichen des gerechten Einsatzes von Materialien und Fertigungsverfahren konnte ich neue Erfahrungen sammeln. Ich erkannte, wie wichtig die Zusammenarbeit im Team ist, um unnötigen Aufwand und allfällige Komplikationen zu vermeiden. Die anschliessende Realisierung des Funktionsmusters erwies sich als interessante Abwechslung zur computerlastigen Konstruktion. Unser Lösungsansatz konnte teilweise erfolgreich getestet werden, was mir viel Freude bereitet und mir bestätigt, dass die intensive Arbeit in den vergangenen Monaten, sich gelohnt hat.
\newline

Mit dem Gesamtergebnis dieser Bachelorarbeit bin ich zufrieden, obwohl einzelne Teilfunktionen nicht wie geplant funktionieren. Die formulierten Massnahmen, welche das Funktionsmuster ergänzen sollen, stimmen mich optimistisch, dass die Umsetzung des Pflanzroboters realistisch ist.
\newline

Rückblickend gefiel mir das vertiefte Arbeiten an einer Problemstellung und die selbständige Arbeitsweise sehr. Der straffe Zeitplan erforderte eine gute Planung und ein erhöhtes Mass an Eigendisziplin. Ich bin der Meinung, dass wir ein gutes Team bildeten und die Kommunikation miteinander funktionierte. Der kollegiale und respektvolle Umgang sehe ich dabei als essentielle Grundlage. Die frühzeitigen Erkenntnisse durch die Funktionsnachweise, erschient mir im Nachhinein als besonders zentral. Dadurch wurde gewährleistet, dass Probleme nicht erst am Funktionsmuster erkannt wurden, sondern bereits zu einem frühen Zeitpunkt.
\newline
Zufrieden blicke ich auf eine intensive Bachelorarbeit zurück. Ich bin optimistisch, dass ich die gemachten Erfahrungen in der Industrie anwenden und davon profitieren kann.
\newline

\textbf{Elektrotechnik - Patrick Rossacher}
\newline
\textit{(pro)} Das Projekt Planting Robot hat mich von Anfang an begeistert. Die Möglichkeit eine interdisziplinäre Projektarbeit, von der Ausarbeitung des Pflichtenhefts bis hin zur Testphase der Gesamtlösung, durchführen zu können ist eine seltene Gelegenheit. Dabei konnte ich als Elektrotechnik Student auch massgeblich bei der Konzeptionierung der Maschinenbauteile mitarbeiten. Alle wichtigen Projektentscheidungen wurden stets im Team gefällt, was die Zusammenarbeit sehr interessant gestaltete. Weiter hatten wir eine fast uneingeschränkte Freiheit bei der Designfindung. Dies führte zu vielen kreativen Ideen und zu einer interessanten Arbeit.
\newline
	
Das Projekt verlief in den Hauptphasen Konzept, Realisierung und Inbetriebnahme sehr unterschiedlich. Der kreative Teil der Lösungsfindung zu Anfang des Projekts nahm viel Zeit in Anspruch. Die Zusammenarbeit im Team war sehr eng, trotzdem locker und ausgelassen. Auch wurden die betreuenden Dozent stark in den Prozess miteinbezogen. So wurden wöchentlich abgehaltene Diskussionsrunden zu viert, zu einem wichtigen Bestandteil dieses Projektabschnitts. Der Wechsel zur Phase der Realisierung teilte das Team in die beiden Lager der jeweiligen Fachdisziplin. Der Output wurde dabei deutlich gesteigert, allerdings wurden die Informationswege länger und teils kleinere Probleme bei Schnittstellenkomponenten entstanden. Der Zeitdruck entwickelte sich während dieser Zeit zum Hauptproblem des Projektes. Die Arbeit wurde immer hastiger und der Fokus verschob sich von einer innovativen Ideenfindung hin zu pragmatischeren Ansätzen. Die letzte Phase der Inbetriebnahme und des Testings im Team, konnte leider nur noch sehr kurz und nicht abschliessend durchgeführt werden. Die Tatsache des Zeitmangels ist hier besonders bedauerlich, da es sich meiner Meinung nach um eine der interessanteste Phase des Projekts handelte.
\newline

Trotz des erwähnten Zeitmangels konnte ein grosser Teil der im Pflichtenheft definierten Anforderungen erfüllt, oder die Voraussetzungen zur Erfüllung dieser Punkte geschaffen werden. Mit dieser Arbeit konnte gezeigt werden, dass die Verwendung von Nemacaps in einem industriellen Produktionsprozess für Topfpflanzen möglich ist.
\newline

\textbf{Danksagung}
\newline
\textit{(ygu/pro)} Zum Schluss möchten wir allen beteiligten Personen danken, welche uns während dieser Zeit unterstützt haben. Ein besonderer Dank geht an unsere betreuenden Dozenten, Markus Thalmann und Marco De Angelis. Dank Ihrer unkomplizierten und kompetenten Beratung waren Sie stets eine Hilfe. Auch möchten wir uns bei Herrn Jean-Antoine Meiners für die Ermöglichung dieser Bachelorarbeit bedanken. Das grosse Vertrauen das er in uns hatte und die stete Motivation, schätzten wir sehr.

