\newpage
\section{Schlusswort}

\textbf{Maschinentechnik - Yves Gubelmann}
\newline
\textit{(ygu)} Die Bachelorarbeit begleitete mich durch das sechste und letzte Semester im Studium Maschinentechnik an der Hochschule Luzern. Die erhaltene Aufgabenstellung schien mir zu Beginn leicht abstrakt und nur grob definiert. Diese offene Formulierung der Aufgabe gab uns ein hohes Mass an Gestaltungsfreiheit sowie auch Eigenverantwortung. Dies bedeutete, dass wir in der Entwurfsphase eine aktive Rolle im Setzen der Rahmenbedingungen einnahmen, was ich schätzte. Der stetige Austausch mit dem Industriepartner sowie den betreuenden Dozenten in dieser Phase führte zu einem klaren Verständnis der Aufgabenstellung sowie Vorgehensweise.
\newline

Als spannende Phase der Bachelorarbeit empfand ich die Konzeptausarbeitung. Dabei entstanden unzählige interessante Lösungen, welche nur durch den interdisziplinären Austauch miteinander möglich wurden. Die kreative Findung von innovativen Lösungen war bereichernd, jedoch auch entscheidend für den weiteren Verlauf dieser Arbeit. Nur eine sorgfältige Abwägung aller Kriterien gewährleistet eine pragmatische Wahl des Lösungskonzeptes.
\newline

Die Umsetzungsphase bleibt mir persönlich als intensiv sowie auch lernreich in Erinnerung. Die Konstruktion des Funktionsmusters stellte mich vor Herausforderungen die für mich als gelernter Elektroniker unbekannt waren. Gerade den gerechten Einsatz von Materialien sowie Fertigungsverfahren empfand ich als speziell lernreich. Auch erkannte ich, wie wichtig die Zusammenarbeit im Team ist, um unnötiger Aufwand und allfällige Komplikationen zu vermeiden. Die darauffolgende Realisierung des Funktionsmusters ergab einen spannenden Abschluss der theorielastigen Konstruktion. Das erfolgreiche Funktionieren der umgesetzten Lösung bereitete mir viel Freude und zeigte, dass die monatelange Arbeit sich lohnte.
\newline

Das Gesamtergebnis dieser Bachelorarbeit stimmt mich zufrieden, auch wenn einzelne Teilfunktionen nicht wie geplant funktionieren. Aufgrund der formulierten Massnahmen bin ich optimistisch, dass dieses Funktionsmuster ergänzt und erfolgreich umgesetzt werden kann.
\newline

Rückblickend gefiel mir das vertiefte Arbeiten an einer Problemstellung sowie die selbständige Arbeitsweise sehr. Der sportlich gesetzte Zeitplan erforderte eine gute Planung und ein erhöhtes Mass an Eigendisziplin. Ich bin der Meinung, dass wir ein gutes Team bildeten und die Kommunikation miteinander funktionierte. Der kollegiale und respektvolle Umgang erschien mir dabei als zentrale Grundlage dafür. Auch erkenne ich, dass der Funktionsnachweis von kritischen Funktionen möglichst frühzeitig und umfassend erfolgen soll. So wird gewährleistet, dass Probleme nicht erst am Funktionsmuster erkannt werden. Auch lernte ich während der Bachelorarbeit, dass die enge Zusammenarbeit mit dem Kunden (hier Industriepartner) von Beginn an, den Fortschritt des Projektes stark unterstützt.
\newline

Zufrieden und mit positiven Erfahrungen blicke ich auf eine intensive Bachelorarbeit zurück. Sie begleitete mich eng durch meinen Abschluss meines Studiums. Ich bin optimistisch, dass ich die gemachten Erfahrungen in der Industrie anwenden und davon profitieren kann.
\newline

\textbf{Elektrotechnik - Patrick Rossacher}
\newline
\textit{(pro)}
\newline	

\textbf{Danksagung}
\newline
\textit{(ygu/pro)} Zum Schluss möchten wir allen beteiligten Personen danken, welche uns in dieser Zeit unterstützt haben. Im Speziellen gilt der Dank unseren betreuenden Dozenten, Markus Thalmann und Marco De Angelis. Dank ihrer unkomplizierten und kompetenten Beratung waren Sie stets eine Hilfe. Auch möchten wir uns bei Herrn Jean-Antoine Meiners für die Ermöglichung dieser Bachelorarbeit bedanken. Das grosse Vertrauen das er in uns hatte und die stete Motivation schätzten wir sehr.

