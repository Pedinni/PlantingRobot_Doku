\newpage
\section{Schlusswort}

\textbf{Maschinentechnik - Yves Gubelmann}
\newline
\textit{(ygu)} Die Bachelorarbeit begleitete mich durch das sechste und letzte Semester im Studium Maschinentechnik an der Hochschule Luzern. Die erhaltene Aufgabenstellung schien mir zu Beginn leicht abstrakt und nur grob definiert. Die offene Formulierung der Aufgabenstellung liess uns einen grossen Gestaltungsfreiraum  zu und konfrontierte uns gleichzeitig mit Eigenverantwortung. Während der Entwurfsphase nahmen wir durch das Festlegen der Rahmenbedingungen eine aktive Rolle ein, was ich sehr schätzte. Der stetige Austausch mit dem Industriepartner und den betreuenden Dozenten führte in dieser Phase zu einem klaren Verständnis der Aufgabenstellung sowie Vorgehensweise.
\newline

Die Konzeptausarbeitung empfand ich als spannender Prozess während der gesamten Bachelorarbeit. Es entstanden unzählige interessante Lösungsansätze, welche nur im interdisziplinären Austausch mit meinem Arbeitskollgen möglich wurden . Die kreative Findung von innovativen Lösungen war bereichernd, jedoch auch entscheidend für den weiteren Verlauf dieser Arbeit. Nur eine sorgfältige Abwägung aller Kriterien gewährleistete eine pragmatische Wahl des Lösungskonzeptes.
\newline

Die Umsetzungsphase bleibt mir als intensiv sowie und lernreich in Erinnerung. Die Konstruktion des Funktionsmusters stellte mich vor Herausforderungen, die für mich als gelernter Elektroniker unbekannt waren. Vor allem in den Bereichen des gerechten Einsatzes von Materialien und Fertigungsverfahren konnte ich neue Erfahrungen sammeln. Ich erkannte, wie wichtig die Zusammenarbeit im Team ist, um unnötigen Aufwand und allfällige Komplikationen zu vermeiden. Die anschliessende Realisierung des Funktionsmusters erwies sich als spannende Abwechslung zur computerlastigen Konstruktion. 
Unserer Lösungsansatz konnte erfolgreich realisiert werden, was mir viel Freude bereitete und ich als gerechten Lohn für die monatelange Arbeit empfand.
\newline

Mit dem Gesamtergebnis dieser Bachelorarbeit bin ich zufrieden, obwohl einzelne Teilfunktionen nicht wie geplant funktionieren. Die formulierten Massnahmen, welche das Funktionsmuster ergänzen sollen, stimmen mich optimistisch. Durch diese Ergänzungen ist unser Pflanzroboter für den Einsatz gewappnet.
\newline

Rückblickend gefiel mir das vertiefte Arbeiten an einer Problemstellung und die selbständige Arbeitsweise sehr. Der sportlich gesetzte Zeitplan erforderte eine gute Planung und ein erhöhtes Mass an Eigendisziplin. Ich bin der Meinung, dass wir ein gutes Team bildeten und die Kommunikation miteinander funktionierte. Der kollegiale und respektvolle Umgang erschien mir dabei als zentrale Grundlage. Das frühzeitige Erkennen der Funktionsnachweise von kritischen Funktionen erschien mir im Nachhinein als besonders zentral. Dadurch wurde gewährleistet, dass Probleme nicht erst am Funktionsmuster erkannt wurden, sondern bereits zu einem frühen Stadium des Projekts. Die enge Zusammenarbeit mit dem Auftraggeber erschien mir als wichtiger Faktor für die Unterstützung des laufenden Fortschritts des Projekts.
\newline

Zufrieden und mit positiven Erfahrungen blicke ich auf eine intensive Bachelorarbeit zurück. Sie begleitete mich eng während dem Abschluss meines Studiums. Ich bin optimistisch, dass ich die gemachten Erfahrungen in der Industrie anwenden und davon profitieren kann.
\newline

\textbf{Elektrotechnik - Patrick Rossacher}
\newline
\textit{(pro)}
\newline	

\textbf{Danksagung}
\newline
\textit{(ygu/pro)} Zum Schluss möchten wir allen beteiligten Personen danken, welche uns während dieser Zeit unterstützt haben. Ein besonderer Dank geht an unseren betreuenden Dozenten, Markus Thalmann und Marco De Angelis. Dank ihrer unkomplizierten und kompetenten Beratung waren Sie stets eine Hilfe. Auch möchten wir uns bei Herrn Jean-Antoine Meiners für die Ermöglichung dieser Bachelorarbeit bedanken. Das grosse Vertrauen das er in uns hatte und die stete Motivation schätzten wir sehr.

