\subsection{Rückblick}
\newpage
\subsubsection{Projektplan}
\label{projektplan}
\textit{ygu} Der zu Beginn erstellte Projektplan wurde im Verlauf des Semesters stetig abgeglichen. Dabei wurden die benötigte Zeit für die einzelnen Arbeitsschritte (Work-Packages) im Projektplan eingetragen. So kann nun am Ende der Bachelorarbeit kritisch reflektiert werden. Die Projektpläne sind im Anhang zu finden.
\newline

\textbf{Maschinentechnik - Yves Gubelmann}
\newline
Für eine  Reflektion eignet sich eine Beurteilung der einzelnen Phasen:
\begin{itemize}
	\item \textbf{Entwurf:} Die Entwurfsphase konnte plangemäss umgesetzt werden. Alle Arbeitsschritte erfolgten in der gesetzten Zeit.
	
	\item \textbf{Konzept:} Während der Konzeptphase konnten auch alle Arbeitsschritte nach Plan einghalten werden. Am wichtigen Meilenstein vom 22.3.17 konnte drei Konzepte und der Entscheid für Konzept Blau vorgestellt werden.
	
	\item \textbf{Umsetzung:} Die Umsetzungsphase startete wie vorgesehen. Dabei wurde während der Konstruktion des Konzepts (work-package 3.2) die Reservezeit benötigt um diesen Arbeitsschritt zu beenden. Somit verzögerten sich die Arbeitschritte 'Fertigungsunterlagen erstellen' und 'Fertigung der Bauteile' um eine Woche. Verzögerungen in Zahlungs- sowie Lieferfristen behinderten die Beschaffung der Komponenten und somit den Fortschritt der Bachelorarbeit um mindestens 10 Arbeitstage, sodass die Inbetriebnahme mit einer Verspätung von eineinhalb Wochen startete. Als positiver Nebeneffekt konnte in diesem Zeitraum ausführlich an der Dokumentation gearbeitet werden.
	
	\item \textbf{Inbetriebnahme}: Der Übergang in diese Phase war unscharf, da die Lieferungen der Komponenten unterschiedlich eintrafen. Dadurch konnte die Montage teilweise ausgeführt werden, wobei Montage des Systems direkt abhängig war von der Lieferung von Blexon GmbH. Für den Teil Maschinentechnik reichte die verbleibende Zeit aus, um die Montage fertigzustellen. Trotzdem wäre ein früherer Start der Inbetriebnahme (um circa zwei Wochen) für das Gesamtresultat positiv gewesen. Schlussendlich fehlten ziemlich genau zwei Wochen, die durch die Komplikationen in Zahlung- und Lieferungstechnisch auftraten.

	
\end{itemize} 

\subsubsection{Risikoanalyse}

\subsubsection{Finanzierung}
\label{finanzierung}