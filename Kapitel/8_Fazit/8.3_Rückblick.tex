\subsection{Rückblick}
Dieses Unterkapitel widmet sich dem Rückblick der vorliegenden Bachelorarbeit. Mit einem Vergleich des Projektplans wird auf den tatsächlichen Verlauf des Projekts zurückgeblickt. Weiter wird auf das Risikomanagement sowie die Finanzierung eingegangen.
\subsubsection{Projektplan}
\label{projektplan}
Der zu Beginn erstellte Projektplan wurde im Verlauf des Semesters stetig abgeglichen. Dabei wurden die benötigte Zeit für die einzelnen Arbeitsschritte (Work-Packages) im Projektplan eingetragen. So kann nun am Ende der Bachelorarbeit kritisch reflektiert werden. Die Projektpläne sind im Anhang zu finden.
\newline

Für die  Reflexion eignet sich eine Beurteilung der einzelnen Phasen:
\begin{itemize}
	\item \textbf{Entwurf:} Die Entwurfsphase konnte plangemäss umgesetzt werden. Alle Arbeitsschritte erfolgten in der gesetzten Zeit.
	
	\item \textbf{Konzept:} Während der Konzeptphase konnten auch alle Arbeitsschritte nach Plan einghalten werden. Am wichtigen Meilenstein vom 22.3.17 konnte drei Konzepte und der Entscheid für Konzept Blau vorgestellt werden.
	
	\item \textbf{Umsetzung:} Die Umsetzungsphase startete wie vorgesehen. In der Maschinentechnik wurde während der Konstruktion des Konzepts (Work-Package 3.2) die Reservezeit benötigt um diesen Arbeitsschritt zu beenden. Somit verzögerten sich die Arbeitschritte 'Fertigungsunterlagen erstellen' und 'Fertigung der Bauteile' um eine Woche. Verzögerungen in Zahlungs- sowie Lieferfristen behinderten die Beschaffung der Komponenten und somit den Fortschritt der Bachelorarbeit um mindestens 10 Arbeitstage, sodass die Inbetriebnahme mit einer Verspätung von eineinhalb Wochen startete. 
	Als positiver Nebeneffekt konnte in diesem Zeitraum ausführlich an der Dokumentation gearbeitet werden.
	\newline
	
	In der Elektrotechnik began die Umsetzungsphase mit der Evaluation der Komponenten. Diese Phase wurde mit einer Verspätung von 9 Arbeitstagen abgeschlossen, wodurch sich die Das Design der Schemas sowie Layouts der PCBs verzögerten. mit einer gleichbleibenden Verzögerung konnte die Implementation der Software erst verspätet gestartet werden.
	
	\item \textbf{Inbetriebnahme}: Der Übergang in diese Phase war unscharf, da die Lieferungen der Komponenten unterschiedlich eintrafen. Dadurch konnte die Montage teilweise ausgeführt werden, wobei Montage des Systems direkt abhängig war von der Lieferung von Blexon GmbH. Für den Teil Maschinentechnik reichte die verbleibende Zeit aus, um die Montage fertigzustellen. Trotzdem wäre ein früherer Start der Inbetriebnahme (um circa zwei Wochen) für das Gesamtresultat positiv gewesen. der Elektrotechnik fehlte schlussendlich ziemlich genau die zwei Wochen, die durch die Komplikationen in Zahlung- und Lieferungstechnisch auftraten.

	
\end{itemize} 

\subsubsection{Risikomanagement}
\textit{(pro)} In diesem Kapitel werden die im Dokument Risikomanagement aufgeführten Risiken reflektiert. Im folgenden Abschnitt wird auf die Risiken eingegangen welche so, oder in ähnlicher Form eingetreten sind:

\begin{itemize}
	\item \textbf{Zu hohe Kosten:} Weil die dem Industriepartner unterbreiteten Offerten höher ausfielen als zuvor Besprochen, konnte das Geld vom Industriepartner nicht sofort überwiesen werden.
	\item \textbf{Lieferzeit bestellter Komponenten wird nicht eingehalten:} Die im vorherigen Punkt genannte Verzögerung wirkte sich direkt auf die Komponenten Bestellung aus. Dadurch entstand eine Lieferverzögerung von zwei Wochen.
	\item \textbf{Unerwartete Komplikationen bei Inbetriebnahme:} Durch stetige Verzögerungen während der Projektarbeit, konnte der Planting Robot nicht als Gesamtlösung getestet werden. Aufgrund der fehlenden Testergebnisse kann die Funktion der Anlage nicht verifiziert werden.
\end{itemize}

Die erwähnte Lieferverzögerung konnte durch vorziehen anderer Arbeitspaket wie Dokumentation, Software sowie konstruieren von weiteren Baugruppen grösstenteils überbrückt werden. Die fehlende Zeit in der Testphase konnte jedoch nicht mehr aufgeholt werden. Somit wurden für dieses Risiko keine Massnahmen ergriffen und es ist in vollem Ausmass eingetreten.

\subsubsection{Finanzierung} \label{finanzierung}
Dieses Unterkapitel bietet eine Übersicht über die getätigten Ausgaben. Die genaue Auflistung aller Positionen sämtlicher Bestellungen findet sich im Anhang. Nach folgendem Schema wurde bei einer grösseren Materialbestellung vorgegangen:

\begin{enumerate}
	\item Die zu bestellenden Komponenten wurden in einer Offerte zusammengefasst und an unseren Industriepartner gesendet.
	\item Unser Industriepartner überwies den Betrag der Offerte auf unser Konto.
	\item Nach Erhalt der Zahlung lösten wir die Bestellungen bei unseren Lieferanten aus.
\end{enumerate}


In Tabelle \ref{tab:Finanzierung_Offerte1} und \ref{tab:Finanzierung_Offerte2} sind alle Beträge aufgeführt, welche in Zusammenhang mit den gestellten Offerten an MCC Laboratoire Meiners verrechnet wurden. 

\begin{table}[H]
	\small
	\centering
	\caption{Finanzierung, Abrechnung 1. Offerte}
	\begin{tabular}{lrlr}
		\hline
		\multicolumn{1}{|l|}{\textbf{Datum}} & \multicolumn{1}{l|}{\textbf{Partei}} & \multicolumn{1}{l|}{\textbf{Bemerkung}} & \multicolumn{1}{l|}{\textbf{Betrag [CHF]}} \\
		\hline
		\multicolumn{1}{|l|}{12.04.2017} & \multicolumn{1}{l|}{Rossacher} & \multicolumn{1}{l|}{Adafruit} & \multicolumn{1}{r|}{-97.95} \\
		\hline
		\multicolumn{1}{|l|}{18.04.2017} & \multicolumn{1}{l|}{Meiners} & \multicolumn{1}{l|}{1. Überweisung} & \multicolumn{1}{r|}{1'372.10} \\
		\hline
		\multicolumn{1}{|l|}{19.04.2017} & \multicolumn{1}{l|}{Rossacher} & \multicolumn{1}{l|}{Adafruit Zoll} & \multicolumn{1}{r|}{-27.90} \\
		\hline
		\multicolumn{1}{|l|}{18.04.2017} & \multicolumn{1}{l|}{Rossacher} & \multicolumn{1}{l|}{Pololu} & \multicolumn{1}{r|}{-212.20} \\
		\hline
		\multicolumn{1}{|l|}{18.04.2017} & \multicolumn{1}{l|}{Rossacher} & \multicolumn{1}{l|}{Distrelec} & \multicolumn{1}{r|}{-578.85} \\
		\hline
		\multicolumn{1}{|l|}{19.04.2017} & \multicolumn{1}{l|}{Gubelmann} & \multicolumn{1}{l|}{Mädler} & \multicolumn{1}{r|}{-253.40} \\
		\hline
		\multicolumn{1}{|l|}{24.04.2017} & \multicolumn{1}{l|}{Gubelmann} & \multicolumn{1}{l|}{Ringspann} & \multicolumn{1}{r|}{-89.86} \\
		\hline
		\multicolumn{1}{|l|}{25.04.2017} & \multicolumn{1}{l|}{Rossacher} & \multicolumn{1}{l|}{Pololu Zoll} & \multicolumn{1}{r|}{-40.40} \\
		\hline
		\multicolumn{1}{|l|}{03.05.2017} & \multicolumn{1}{l|}{Gubelmann} & \multicolumn{1}{l|}{igus} & \multicolumn{1}{r|}{-115.91} \\
		\hline
		&       & \textbf{Bilanz} & \textbf{-44.37} \\
	\end{tabular}%
	\label{tab:Finanzierung_Offerte1}%
\end{table}%


\begin{table}[H]
	\small
	\centering
	\caption{Finanzierung, Abrechnung 2. Offerte}
	\begin{tabular}{lrlr}
		\hline
		\multicolumn{1}{|l|}{\textbf{Datum}} & \multicolumn{1}{l|}{\textbf{Partei}} & \multicolumn{1}{l|}{\textbf{Bemerkung}} & \multicolumn{1}{l|}{\textbf{Betrag [CHF]}} \\
		\hline
		\multicolumn{1}{|l|}{08.05.2017} & \multicolumn{1}{l|}{Meiners} & \multicolumn{1}{l|}{2. Überweisung} & \multicolumn{1}{r|}{1'302.25} \\
		\hline
		\multicolumn{1}{|l|}{08.05.2017} & \multicolumn{1}{l|}{Gubelmann} & \multicolumn{1}{l|}{Kanya} & \multicolumn{1}{r|}{-700.80} \\
		\hline
		\multicolumn{1}{|l|}{19.05.2017} & \multicolumn{1}{l|}{Gubelmann} & \multicolumn{1}{l|}{blexon} & \multicolumn{1}{r|}{-408.05} \\
		\hline
		\multicolumn{1}{|l|}{23.05.2017} & \multicolumn{1}{l|}{Rossacher} & \multicolumn{1}{l|}{Digikey} & \multicolumn{1}{r|}{-168.10} \\
		\hline
		\multicolumn{1}{|l|}{26.05.2017} & \multicolumn{1}{l|}{Gubelmann} & \multicolumn{1}{l|}{Landi, Coop} & \multicolumn{1}{r|}{-60.90} \\
		\hline
		&       & \textbf{Bilanz} & \textbf{-35.60} \\
	\end{tabular}%
	\label{tab:Finanzierung_Offerte2}%
\end{table}%

Die Differenz aus der Offerte und den effektiven Bestellwerten ist jeweils unter der Position Bilanz aufgeführt.\\
Des weiteren stehen jedem Elektrotechnik Studenten während der BDA CHF 500.- Budget von der Hochschule Luzern zur Verfügung. Kleinbeträge für Elektronikkomponenten konnten so direkt über die Hochschule bestellt werden. Das Hochschulbudget ist in Tabelle \ref{tab:Finanzierung_HSLU_Budget} aufgeführt.

\begin{table}[H]
	\small
	\centering
	\caption{Finanzierung, Abrechnung HSLU Budget}
	\begin{tabular}{lrlr}
		\hline
		\multicolumn{1}{|l|}{\textbf{Datum}} & \multicolumn{1}{l|}{\textbf{Partei}} & \multicolumn{1}{l|}{\textbf{Bemerkung}} & \multicolumn{1}{l|}{\textbf{Betrag [CHF]}} \\
		\hline
		\multicolumn{1}{|l|}{} & \multicolumn{1}{l|}{HSLU} & \multicolumn{1}{l|}{Budget} & \multicolumn{1}{r|}{500.00} \\
		\hline
		\multicolumn{1}{|l|}{12.04.2017} & \multicolumn{1}{l|}{Rossacher} & \multicolumn{1}{l|}{1. Bestellung} & \multicolumn{1}{r|}{-185.70} \\
		\hline
		\multicolumn{1}{|l|}{03.05.2017} & \multicolumn{1}{l|}{Rossacher} & \multicolumn{1}{l|}{2. Bestellung} & \multicolumn{1}{r|}{-140.02} \\
		\hline
		\multicolumn{1}{|l|}{09.05.2017} & \multicolumn{1}{l|}{Rossacher} & \multicolumn{1}{l|}{3. Bestellung} & \multicolumn{1}{r|}{-37.10} \\
		\hline
		&       & \textbf{Bilanz} & \textbf{137.18} \\
	\end{tabular}%
	\label{tab:Finanzierung_HSLU_Budget}%
\end{table}%

Die jeweiligen Bilanzen der vorangehenden Tabellen werden in Tabelle \ref{tab:Finanzierung_Bilanz} zusammengetragen. Der dadurch entstandene Restbetrag von CHF 57.21 ist der Firma MCC Labarotoire Meiners gutzuschreiben.

\begin{table}[H]
	\small
	\centering
	\caption{Finanzierung, Bilanz}
	\begin{tabular}{lrlr}
		\hline
		\multicolumn{1}{|l|}{\textbf{Datum}} & \multicolumn{1}{l|}{\textbf{Partei}} & \multicolumn{1}{l|}{\textbf{Bemerkung}} & \multicolumn{1}{l|}{\textbf{Betrag [CHF]}} \\
		\hline
		\multicolumn{1}{|l|}{03.05.2017} & \multicolumn{1}{l|}{Meiners} & \multicolumn{1}{l|}{1. Offerte} & \multicolumn{1}{r|}{-44.37} \\
		\hline
		\multicolumn{1}{|l|}{09.05.2017} & \multicolumn{1}{l|}{HSLU} & \multicolumn{1}{l|}{HSLU Budget} & \multicolumn{1}{r|}{137.18} \\
		\hline
		\multicolumn{1}{|l|}{26.05.2017} & \multicolumn{1}{l|}{Meiners} & \multicolumn{1}{l|}{2. Offerte} & \multicolumn{1}{r|}{-35.6} \\
		\hline
		&       & \textbf{Restbetrag} & \textbf{57.21} \\
	\end{tabular}%
	\label{tab:Finanzierung_Bilanz}%
\end{table}%


