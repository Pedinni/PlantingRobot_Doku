\subsection{Zielbezug}
Der Zielbezug wird anhand des verfassten Pflichtenhefts durchgeführt. Die Orientierung am Pflichtenheft gewährleistet eine objektive Beurteilung der gesetzten Ziele. Dabei stützt sich die Beurteilung der einzelnen Ziele auf den gewonnen Erkenntnisse aus Kapitel \ref{inbetriebnahme}.
\newline

Die Verifikation der gesetzten Ziele ist in Tab. \ref{tab:verifikation} aufgelistet. Diese Tabelle bezieht sich auf das verfasste Pflichtenheft (vgl. Anhang: \textit{Pflichtenheft v1.6}). Dabei sind in dieser Tabelle alle relevanten Punkte für den Zielbezug zusammengefasst.

\newpage
\begin{table}[H]
	\begin{tabular}{|L{0.5cm}|L{10.3 cm}|C{1.3cm}|C{2.3cm}|}
	\hline 
	\textbf{Nr.} & \textbf{Beschreibung} & \textbf{erfüllt?} & \textbf{Kommentar} \\ 
	\hline 
	1.2 &\textbf{Zielgruppe:} Gemüse- und Zierpflanzenbau. Töpfe mit Nenndurchmesser 
	90mm bis 140mm  & Ja & \\ 
	\hline 
	2.1 & \textbf{Abmessungen NemaCaps:} \newline 3mm (Durchmesser), +0.6mm  & Ja & Handhabung erfüllt \\ 
	\hline 
	2.2 & \textbf{Beschaffenheit NemaCaps:} \newline Elastisch, Widerstandsfähig & Ja & Handhabung erfüllt \\ 
	\hline 
	2.3 & \textbf{Handhabung:} In geschlossenem 
	Behälter unter Beigabe eines hygrophoben Pulver, ohne Zugabe von Wasser & Ja &  \\ 
	\hline 
	4.1 & \textbf{Aufbau Robot:} Stationär, auf Boden, an Topfmaschine fixiert & Ja &  \\ 
	\hline 
	4.2 & \textbf{Eingriffsort:} An Topfkranz, vor der Umleitung auf das 
	Förderband. & Ja &  \\ 
	\hline 
	4.3 & \textbf{Lagerung der NemaCaps:} Min. 10‘000 Stück Lagerkapazität, nicht länger als 1 Tag  & Ja &  \\ 
	\hline 
	4.4 & \textbf{Speisung:} 230V Netzspannung, max. Leistungsaufnahme 2kW & Ja &  \\ 
	\hline 
	4.5 & \textbf{Topferkennung:} Setzprozess soll nur ausgeführt werden, wenn sich 
	ein Topf auf dem Topfkranz befindet.  & unklar & Verifikation offen \\ 
	\hline 
	4.6 & \textbf{Topfkonfiguration (Fest):} \newline Der Planting Robot muss für jeden Batch mit der 
	Topfgrösse konfiguriert werden.  & Ja & Wunsch-anforderung umgesetzt \\ 
	\hline 
	4.7 & \textbf{Topfkonfiguration (Wunsch):} Topfgrösse selbständig erkennen und dementsprechend konfigurieren.  & Ja & Konfiguration manuell möglich \\ 
	\hline 
	5.1 & \textbf{Einsetztiefe:} variabel, einstellbar. 
	Maximale Einsetztiefe: 60\% der Topfhöhe & bedingt & nicht implementiert \\ 
	\hline 
	5.2 & \textbf{Bruchverhalten NemaCaps:} Dürfen nach Setzprozess beschädigt sein. Alle 
	Bestandteile müssen sich in der Erde befinden.  & unklar & Verifikation offen \\ 
	\hline 
	5.3 & \textbf{Einsetzlokalität:} 3 Stück um den Mittelpunkt zu 120° versetzt, mit Durchmesser von 60\% des 
	Topfdurchmessers, NemaCaps dürfen nicht durch das Loch für die Stecklinge 
	eingesetzt werden.  & Ja &  \\ 
	\hline 
	5.4 & \textbf{Eingriffszeitpunkt:} Während Stopp Phase. Verhältnis 
	Bewegungszeit/Eingriffszeit = 1:1. & unklar & Verifikation offen \\ 
	\hline 
	5.7 &  \textbf{Eingriffszeit 
	bei normaler Auslastung:} \newline 0.64s (2800 Töpfe/Stunde) & unklar & Verifikation offen \\ 
	\hline 
	5.8 &  \textbf{Eingriffszeit 
	bei maximaler Auslastung:} \newline 0.5s (3600 Töpfe/Stunde) & unklar & Verifikation offen \\ 
	\hline 
	\end{tabular} 
	\caption{Verifikation der gesetzten Ziele durch das Pflichtenheft}
	\label{tab:verifikation}
\end{table}	
\newpage
Ergänzend ist hinzuzufügen:
\begin{itemize}

	\item Die Handhabung der NemaCaps (Punkte 2.1 und 2.2) ist mit dem umgesetzten Funktionsmuster gewährleistet. Für die erfolgreiche Handhabung ist es essentiell, dass ausschliesslich frische NemaCaps verwendet werden. 
	
	\item Die variable Einstellung der Einsetztiefe ist durch die Konstruktion sowie auch den Antrieb technisch möglich (Punkt 5.1). Die Implementation in der Software fehlt, wodurch dieser Punkt nur \textit{bedingt} erfüllt ist.
	
	\item Die Punkte 5.2, 5.4, 5.7 und 5.8 konnten nicht verfiziert werden, da für die Inbetriebnahme der Gesamtfunktion der zeitliche Rahmen fehlte. Da somit das Zusammenspiel der einzelnen Funktion nicht geprüft wurde, kann über diese Punkte keine Aussage gemacht werden.
	
	\item Die Topferkennung (Punkt 4.5) funktioniert prinzipiell. Durch die fehlende Zeit konnte die Implementation an der Topfmaschine (oder Testumgebung) nicht durchgeführt werden.
	
	\item Die Erfüllung von Punkt 4.6 ist obsolet, da die Wunschanforderung realisiert wurde.
	
	\item die selbständige Topfkonfiguration (Punkt 4.7) funktioniert. Die Kopplung mit der automatischen Topferkennung (Punkt 4.5) ist nicht implementiert, zurzeit muss die Konfiguration von einem Benutzer ausgelöst werden.
	
	\item Die Funktionen \textit{NemaCaps vereinzeln} und \textit{Setzmechanismus konfigurieren} funktionieren grundsätzlich. Da die formulierten Punkte im Pflichtenheft sich auf das ultimative Endresultat beziehen, haben diese Teilerfolge nur wenig Gewicht in der Verifikation.
\end{itemize}

Abschliessend wird befunden, dass alle Funktionen, die im zeitlichen Rahmen gestestet wurden, ihre Funktion erfüllen. Für alle anderen (nicht getesteten) kann zum jetzigen Zeitpunkt kein abschliessendes Urteil gemacht werden.