\subsection{Projektplan}
\textit{(ygu)} Vorgängig wurde für beide Fachrichtungen ein Projektplan ausgearbeitet. Dabei konnten die wichtigsten Arbeitsschritte in sogenannte Work-Packages gegliedert werden, die sich an der Methodik aus Kapitel \ref{methodik} orientieren. Nun wurde der Zeitbedarf für jedes Work-Package eingeplant, wobei eine Auflösung von einem Tag als angemessen erschien. Ergänzt wurde der Projektplan mit:

\begin{itemize}
	\item \textbf{Meilensteine}, welche den Abschluss wichtiger Phasen markieren.
	
	\item \textbf{Besprechungen} mit dem betreuenden Dozenten.
	
	\item \textbf{Reservezeit} für Work-Packages die zeitlich schwer kalkulierbar sind. Auch nimmt die Dauer der eingeplanten Reservezeit im Verlauf des Semesters zu.
	
	\item \textbf{Notizen} relevanter Ereignisse.
\end{itemize}

Während dem Verlauf dieser Bachelorarbeit wurde die tatsächlich benötigte Zeit erfasst. Der Vergleich zwischen Planung und praktischer Umsetzung des Projekts folgt im Kapitel  \ref{projektplan}. Die Projektpläne beider Fachrichtungen sind im Anhang ersichtlich. 