\subsection{Projektplan}
\textit{(ygu)} Vorgängig wurde für beide Fachrichtungen ein Projektplan ausgearbeitet. Dabei wurden die wichtigsten Arbeitsschritte in sogenannte Work-Packages gegliedert und orientiert an der Methodik aus Kapitel \ref{methodik} geordnet. Nun wurde der Zeitbedarf für jedes Work-Package eingeplant, wobei eine Auflösung auf einen Tag als angemessen erschien. Ergänzt wurde der Projektplan mit:

\begin{itemize}
	\item \textbf{Meilensteine}, welche den Abschluss wichtiger Phasen markieren.
	
	\item \textbf{Besprechungen} mit dem betreuenden Dozenten.
	
	\item \textbf{Reservezeit} für Work-Packages die zeitlich schwer kalkulierbar sind. Auch nimmt die Dauer der eingeplanten Reservezeit im Verlauf des Semesters zu.
	
	\item \textbf{Notizen} relevanter Ereignisse.
\end{itemize}

Während dem Verlauf dieser Bachelorarbeit wurde die tatsächlich benötigte Zeit erfasst. Dies ermöglicht einen Vergleich zwischen Planung sowie praktischer Umsetzung des Projekts. Dieser Vergleich wird im Kapitel \ref{projektplan} gemacht. Die Projektpläne beider Fachrichtungen sind im Anhang ersichtlich. 