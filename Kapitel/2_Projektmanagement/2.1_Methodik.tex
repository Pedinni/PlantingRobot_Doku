\newpage
\section{Projektmanagement}
\textit{(ygu)} Dieses Kapitel befasst sich mit relevanten Punkten des Projektmanagements. Dabei wird die methodische Vorgehensweise der vorliegenden Bachelorarbeit erläutert. Weiter befasst sich dieses Kapitel mit dem Projektplan, dem Risikomanagement sowie der Finanzierung. 
\subsection{Methodik}
\label{methodik}
\textit{(ygu)} Diese Bachelorarbeit befasst sich mit systematischen Produktentwicklung eines Pflanzroboters. Das systematische Vorgehen orientiert sich  an der VDI-Richtlinie 2221, welche in folgende Phasen unterteilt wird \cite{naefe}:

\begin{itemize}
	\item \textbf{Phase I:} Planen und klären der Aufgabe durch informative Festlegung
	
	\item \textbf{Phase II:} Konzipieren durch prinzipielle Festlegung
	
	\item \textbf{Phase III:} Gestalterische Festlegung der angestrebten Lösung
	
	\item \textbf{Phase IV:} Ausarbeiten, Erstellung der erforderlichen Unterlagen
	
\end{itemize}

Phase I stellt den eigentlichen Entwurf des Systems dar. Gemäss der VDI-Richtlinie 2221 beinhaltet diese Phase im wesentlichen das Klären der Aufgabenstellung und die Ermittlung von Funktionen \cite{vdi2221}. Für die nähere Umschreibung der Aufgabenstellung wurde gemeinsam mit dem Auftraggeber (MCC Laboratoire Meiners GmbH ) ein Pflichtenheft ausgearbeitet. Darin sind alle relevanten Eigenschaften und Anforderungen an das Produkt formuliert. Dieses Dokument dient als Grundlage für die weitere Entwicklung des Produktes und weiter als Abgrenzung für die Aufgabenstellung der HSLU T\&A. Der zweite Teil von Phase I, die Ermittlung der Funktionen, gestaltet sich wie folgt \cite{pahl}:

\begin{itemize}
	\item \textbf{Hauptfunktion:} Auf der Basis der Aufgabenstellung und des erarbeiteten Pflichtenheft ergibt sich die Hauptfunktion des Pflanzroboters.
	
	\item \textbf{Aufgliederung in Teilfunktionen:} Die Hauptfunktion wird mittels Funktionsanalyse in verschiedene Teilfunktionen unterteilt. Dadurch kann die Aufgabenstellung abstrahiert, rein funktional betrachtet und folglich eine möglichst lösungsneutrale Betrachtung des Problems gewährleistet werden.
	
	\item \textbf{Funktionsbezogene Variation:} Für die ausgearbeiteten Teillösungen werden mögliche Lösungsansätze erarbeitet. Dafür werden verschiedene Kreativitätstechniken verwendet, um möglichst umfassende Lösungsansätzen für jede einzelne Teilfunktion zu generieren.

\end{itemize}

Nach  VDI-Richtlinie 2221 befasst sich Phase II mit der Ausarbeitung eines Konzepts. Dabei werden die generierten Ideen aus der funktionsbezogenen Variation aufgegriffen und zu einem Konzept ausgearbeitet. Diese Phase ist wie folgt aufgebaut \cite{vdi2221}:

\begin{itemize}
	\item \textbf{Suchen nach Lösungsprinzipien:} Das Gesamtkonzept ergibt sich durch die Aggregation der Teillösungen. Dafür wurden alle Teillösungen in einem morphologischen Kasten aufgelistet und in einem ersten Schritt fünf Konzepte ausgearbeitet. Durch die Methode des „Ausscheidens und Bevorzugens“ \cite{naefe} wird in einem weiteren Schritt die Auswahl auf drei Konzepte eingegrenzt.
	
	\item \textbf{Gliedern in realisierbare Module:} Die Konzepte wurden durch die Gestaltung von Konzeptskizzen visualisiert und zum ersten Mal als gesamtes Konzept dargestellt. 
	
	\item \textbf{Funktionsnachweis kritischer Funktionen:} Ergänzend zur VDI-Richtlinie 2221 wurde ein Funktionsnachweis von kritischen Funktionen erbracht. Dabei wurden einfach aufgebaute, praktische Versuche der erarbeiteten Teillösung realisiert, um die Funktionalität zu überprüfen.
	
	\item \textbf{Bewertung:} Hinzukommend zur VDI-Richtlinie 2221 wurden die ausgearbeiteten Konzepte einer objektiven Bewertung unterzogen. Dabei wurde eine Nutzwertanalyse durchgeführt \cite{pahl}:
	\begin{itemize}
		\item Es wurden fünf Ziele definiert, die sowohl funktionale, als auch wirtschaftliche Anforderungen beinhalten. Für jedes Ziel wurde eine Gewichtung vorgenommen, wobei die die Funktionserfüllung gegenüber dem Kostenaspekt höher gewertet wurde.
		
		\item Jedes der ausgearbeiteten Konzepte wurde nach VDI-Richtlinie 2225 mit einer Wertskala von 0 bis 4 (von unbefriedigend bis ideal) bewertet \cite{vdi2225}. Dabei wurden auch die gewonnenen Erkenntnisse aus dem Funktionsnachweis miteinbezogen. Die objektive Beurteilung ist für die Betrachtung der ausgearbeiteten Konzepte besonders zentral.
	\end{itemize}	
	
	\item \textbf{Entscheid:} Die Nutzwertanalyse wird ausgewertet und das Konzept mit dem höchsten Nutzwert ausgewählt. Dieses kommt nun in Phase III zur Umsetzung. 
\end{itemize}

In Phase III wird das ausgewählte Konzept fachspezifisch umgesetzt.  Dieser Prozessablauf beinhaltet folgende Tätigkeiten: 
\begin{itemize}
	\item \textbf{Gestalten der Module:} Die Teillösungen des Konzepts werden konkret in Baugruppen realisiert. Dies geschieht auf allen technischen Ebenen (Software, PCB-Design und Konstruktion). Zuerst wurde hierfür eine Evaluation von möglichen Komponenten durchgeführt. Wenn nötig, wurden Berechnungen zur Auswahl der Komponenten gemacht. Dabei ist eine stetige Absprache zwischen der Elektrotechnik und Maschinentechnik essentiell, um mögliche Missverständnisse und Fehler zu vermeiden.
	
	\item \textbf{Gestalten des Produktes:} Dies beinhaltet das konstruktive Zusammenführen aller Baugruppen zu einem Produkt. Fertigungsunterlagen für die verwendeten Fertigungsverfahren wurden erstellt und in Auftrag gegeben. Parallel dazu wurde die Bestellung der Komponenten ausgelöst und Offerten von Herstellern eingeholt. 
\end{itemize}

Die Phase IV umfasst die Ausarbeitung des Produktes. Dabei weicht die vorliegende Bachelorarbeit von der VDI-Richtlinie 2221 ab, indem sich diese Arbeit ausschliesslich der Entwicklung eines Funktionsmusters widmet.
\newline
Die Ausarbeitung in dieser Phase legt den Fokus auf die Montage und Inbetriebnahme des Funktionsmusters. Die einzelnen Baugruppen werden an Funktionstests unterzogen dadurch auf ihre Funktionalität geprüft. Sind alle Baugruppen funktionsfähig, kann das Gesamtsystem getestet werden.
\newline
Während der gesamten Bachelorarbeit werden alle relevanten Schritte und Überlegungen (methodischer sowie technischer Art) dokumentiert und im vorliegenden Bericht festgehalten.