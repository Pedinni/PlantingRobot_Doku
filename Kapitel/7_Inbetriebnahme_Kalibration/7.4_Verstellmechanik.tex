\subsection{Verstellmechanik} \label{sec:Inbetriebnahme_Verstellmechanik}
\textit{(pro)} Weil der für diese Anwendung evaluierte Motor für die Vereinzelung verwendet wurde, wurde hier der Motor mit der kleineren 1:100 Getriebeübersetzung und der höheren Leistung verwendet. Da die Stellgenauigkeit bei dieser Funktion gegenüber der Vereinzelung nicht so kritisch ist, kann der Motorentausch vertreten werden. Die Parametrisierung wurde experimentell ermittelt. Dazu wurde die Einheit von Anschlag zu Anschlag bewegt und die Anzahl Encoder Counts ausgelesen. Die Anzahl der ganzen Bewegung wurde dann linear auf 5 Position, stellvertretend für die 5 verschiedenen Topfgrössen, aufgeteilt. 

\begin{figure}[H]
	\includegraphics[width=1\textwidth]{Illustrationen/7-Inbetriebnahme_und_Kalibration/verstellmechanik_1.jpg}
	\caption{verstellmechanik}
	\label{fig:verstellmechanik}
\end{figure}

Die Verstellmechanik kann durch diese Parametrisierung vorgeführt werden, die einstellbaren Topfgrössen entsprechen jedoch nicht den genauen Massen der verschiedenen Töpfen. In Abb. \ref{fig:verstellmechanik} sind vier verschiedene Stechradien A... D abgebildet welche durch Tastendruck auf dem HMI eingestellt werden können.
\newline
Die Verstellmechanik funktioniert somit prinzipiell einwandfrei und erfüllt die Anforderungen des Pflichtenheftes.