\subsection{Aufgabenstellung}
MCC Laboratoire Meiners GmbH möchte ihr Produkt (NemaCaps) während der halbautomatischen Bestückung von Topfpflanzen in den Töpfen platzieren. Nach der Beschüttung der Erde (Punkt 2 in Abbildung \ref{fig:schema_topfmaschine}) sollen NemaCaps im Wurzelbereich der Pflanze platziert werden. Die manuelle Bestückung von NemaCaps würde einen zusätzlichen personellen Aufwand bedeuten. Ob Gartenbauunternehmen bereit sind diesen aufzuwenden, darf hinterfragt werden. Um den Einsatz von NemaCaps lukrativer zu gestalten, möchte der Auftraggeber ein automatisiertes System entwickeln, welches die Bestückung der NemaCaps ausführt. Das entwickelte System soll flexibel auf verschiedene Topfgrössen anwendbar sein.
\newpage

Diese Bachelorarbeit befasst sich mit der Entwicklung eines Funktionsmusters, welche die automatische Bestückung von NemaCaps an einer Topfmaschine übernimmt. Dieses Funktionsmuster soll eine greifbare Vorstellung vermitteln, wie die industrielle Implementierung dieser Automatisationsaufgabe aussehen kann. 
\newline 

Die Aufgabenstellung beinhaltet:
\begin{itemize}
	\item \textbf{Ausarbeitung eines Pflichtenhefts:} Zu Beginn des Entwicklungsprozesses steht eine genaue Analyse der Aufgabenstellung. In Zusammenarbeit mit MCC Laboratoire Meiners GmbH wird der Umfang der Bachelorarbeit abgegrenzt. Produkteigenschaften, Anforderungen an das Funktionsmuster und Randbedingungen (gegeben durch Topfmaschine, Töpfe und Topferde) stehen dabei im Fokus. Am Ende dieser Phase entsteht ein detailliertes Pflichtenheft.
	 
	\item \textbf{Konzeptausarbeitung und Entscheid:} Auf das Pflichtenheft folgt die interdisziplinäre Ausarbeitung von Lösungsvarianten. Durch eine sorgfältige Auslegung aller Vor- und Nachteile der ausgearbeiteten Lösungsansätze ist die beste Lösungsvariante auszuwählen. Dabei ist eine lösungsneutrale Betrachtung durch die funktionelle Abstraktion besonders zentral. Erst diese schafft die Grundlage für einen kreativen Lösungsfindungsprozess und gewährleistet eine volle Ausschöpfung des Potenzials dieser Phase. Nach der objektiven Beurteilung aller Lösungsvarianten entscheiden sich die Studierenden für eine Auswahl und setzen diese anschliessend um.
	
	\item \textbf{Funktionsnachweis kritischer Funktionen:} Teillösungen, die als kritisch erachtet werden, sind einem praktischen Funktionsnachweis zu unterziehen. Die Erkenntnisse des Funktionsnachweises sollen unterstützend in die Beurteilung der Lösungskonzepte einfliessen und den Entscheid stützen.
	
	\item \textbf{Realisation eines Funktionsmusters:} In der Implementationphase soll das ausgearbeitete Konzept als Funktionsmuster realisiert werden. Dabei werden fachspezifisch die einzelnen Komponenten hergestellt und getestet. 
	
	\item \textbf{Erstellung eines Projektplans:} Die gewünschte Realisation eines Funktionsmusters erfordert einen klar definierten Projektplan. Mehrere Arbeitsschritte werden in Work-Packages zusammengefasst und deren Zeitbedarf eingeplant. Auch  werden Lieferfristen und der erforderliche Zeitbedarf von Fertigungsverfahren im Projektplan mitberücksichtigt. Zudem werden Reservezeiten eingeplant, falls allfällige Verzögerungen eintreten. Optional wird dieser Projektplan durch ein Risikomanagement unterstützt.
	
	\item \textbf{Dokumentation:} Eine umfassende, schriftliche Dokumentation über die methodische Vorgehensweise sowie aller relevanten technischen Überlegungen ist zu verfassen. 
	
\end{itemize}