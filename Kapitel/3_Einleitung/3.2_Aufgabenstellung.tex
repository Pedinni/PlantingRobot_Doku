\subsection{Aufgabenstellung}
MCC Laboratoire Meiners GmbH möchte ihr Produkt (NemaCaps) bei der halbautomatischen Bestückung von Topfpflanzen in den Töpfen platzieren. Nach der Beschüttung der Erde (Punkt 2 in Abbildung \ref{fig:schema_topfmaschine}) sollen NemaCaps im Wurzelbereich der Pflanze platziert werden. Die manuelle Bestückung von NemaCaps birgt einen zusätzlichen personellen Aufwand. Ob Gartenbauunternehmen bereit sind diesen aufzuwenden, darf hinterfragt werden. Um den Einsatz von NemaCaps lukrativer zu gestalten, möchte MCC Laboratoire Meiners GmbH ein automatisiertes System (besser Roboter?) entwickeln, welcher die Bestückung der NemaCaps ausführt (übernimmt?). Dabei soll dieses System auch in der Lage sein, verschiedene Topfgrössen zu handhaben.
\newline

Diese Bachelorarbeit befasst sich mit der Entwicklung eines Funktionsmusters, welche die automatische Bestückung von NemaCaps an einer Topfmaschine übernimmt. Dieses Funktionsmuster soll eine greifbare Vorstellung vermitteln, wie die industrielle Implementierung dieser Automatisationsaufgabe aussehen kann. 
\newline 

Die Aufgabenstellung beinhaltet:
\begin{itemize}
	\item \textbf{Ausarbeitung eines Pflichtenhefts:} Am Anfang vom Entwicklungsprozess steht eine genaue Analyse der Aufgabenstellung. In Zusammenarbeit mit MCC Laboratoire Meiners GmbH soll der Umfang der Bachelorarbeit abgegrenzt werden. Produkteigenschaften, Anforderungen an das Funktionsmuster und Randbedingungen (gegeben durch Topfmaschine, Töpfe und Topferde) stehen dabei im Fokus. Am Ende dieser Phase soll ein detailliertes Pflichtenheft entstehen.
	 
	\item \textbf{Konzeptausarbeitung und Entscheid:} Auf das Pflichtenheft folgt die interdisziplinäre Ausarbeitung von Lösungsvarianten. Durch eine sorgfältige Auslegung aller Vor- und Nachteile der ausgearbeiteten Lösungsvarianten soll die Beste Lösungsvariante ausgeweählt werden. Wichtig ist dabei, dass durch die funktionelle Abstraktion der Aufgabe eine lösungsneutrale Betrachtung möglich wird. Erst dies schafft die Grundlage für einen kreativen Lösungsfindungsprozess und gewährleistet eine volle Auschöpfung des Potenzials dieser Phase. Am Ende dieser Phase, und nach einer objektiven Beurteilung aller Lösungsvarianten, soll ein Entscheid stehen, welches der ausgewählten Lösungsvarianten umgesetzt wird.
	
	\item \textbf{Funktionsnachweis krtischer Funktionen:} Teillösungen, die in für eine Umsetzung in Betracht gezogen werden und Unklarheit herrscht, ob diese technisch umsetzbar ist, sind einem Funktionsnachweis zu unterziehen. Die Erkenntnisse des Funktionsnachweises sollen unterstützend in die Beurteilung der Lösungskonzepte einfliessen und den Entscheid besser legitimieren.
	
	\item \textbf{Realisation eines Funktionsmusters:} In der Implementationphase soll das ausgearbeitete Konzept als Funktionsmuster realisiert werden. In dieser Phase werden fachspezifisch die einzelnen Komponenten hergestellt und getestet. (Mehr nötig?)
	
	\item \textbf{Erstellung eines Projektplans:} Die gewünschte Realisation eines Funktionsmusters erfordert einen klar definierten Zeitplan. Darin sollen auch Lieferfristen und der erforderliche Zeitbedarf von Fertigungsverfahren mitberücksichtigt werden. Auch sollen Reservezeiten eingeplant werden, falls allfällige Verzögerungen eintretten. Optional (Idealerweise?) wird dieser Projektplan von einer Risikoanalyse begleitet (unterstützt?).
	
	\item \textbf{Dokumentation:} Eine umfassende schriftliche Dokumentation über die methodische Vorgehensweise sowie aller relevanten technischen Überlegungen ist zu verfassen. 
	
\end{itemize}